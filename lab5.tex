% Options for packages loaded elsewhere
\PassOptionsToPackage{unicode}{hyperref}
\PassOptionsToPackage{hyphens}{url}
\PassOptionsToPackage{dvipsnames,svgnames,x11names}{xcolor}
%
\documentclass[
  letterpaper,
  DIV=11,
  numbers=noendperiod]{scrartcl}

\usepackage{amsmath,amssymb}
\usepackage{lmodern}
\usepackage{iftex}
\ifPDFTeX
  \usepackage[T1]{fontenc}
  \usepackage[utf8]{inputenc}
  \usepackage{textcomp} % provide euro and other symbols
\else % if luatex or xetex
  \usepackage{unicode-math}
  \defaultfontfeatures{Scale=MatchLowercase}
  \defaultfontfeatures[\rmfamily]{Ligatures=TeX,Scale=1}
\fi
% Use upquote if available, for straight quotes in verbatim environments
\IfFileExists{upquote.sty}{\usepackage{upquote}}{}
\IfFileExists{microtype.sty}{% use microtype if available
  \usepackage[]{microtype}
  \UseMicrotypeSet[protrusion]{basicmath} % disable protrusion for tt fonts
}{}
\makeatletter
\@ifundefined{KOMAClassName}{% if non-KOMA class
  \IfFileExists{parskip.sty}{%
    \usepackage{parskip}
  }{% else
    \setlength{\parindent}{0pt}
    \setlength{\parskip}{6pt plus 2pt minus 1pt}}
}{% if KOMA class
  \KOMAoptions{parskip=half}}
\makeatother
\usepackage{xcolor}
\setlength{\emergencystretch}{3em} % prevent overfull lines
\setcounter{secnumdepth}{5}
% Make \paragraph and \subparagraph free-standing
\ifx\paragraph\undefined\else
  \let\oldparagraph\paragraph
  \renewcommand{\paragraph}[1]{\oldparagraph{#1}\mbox{}}
\fi
\ifx\subparagraph\undefined\else
  \let\oldsubparagraph\subparagraph
  \renewcommand{\subparagraph}[1]{\oldsubparagraph{#1}\mbox{}}
\fi


\providecommand{\tightlist}{%
  \setlength{\itemsep}{0pt}\setlength{\parskip}{0pt}}\usepackage{longtable,booktabs,array}
\usepackage{calc} % for calculating minipage widths
% Correct order of tables after \paragraph or \subparagraph
\usepackage{etoolbox}
\makeatletter
\patchcmd\longtable{\par}{\if@noskipsec\mbox{}\fi\par}{}{}
\makeatother
% Allow footnotes in longtable head/foot
\IfFileExists{footnotehyper.sty}{\usepackage{footnotehyper}}{\usepackage{footnote}}
\makesavenoteenv{longtable}
\usepackage{graphicx}
\makeatletter
\def\maxwidth{\ifdim\Gin@nat@width>\linewidth\linewidth\else\Gin@nat@width\fi}
\def\maxheight{\ifdim\Gin@nat@height>\textheight\textheight\else\Gin@nat@height\fi}
\makeatother
% Scale images if necessary, so that they will not overflow the page
% margins by default, and it is still possible to overwrite the defaults
% using explicit options in \includegraphics[width, height, ...]{}
\setkeys{Gin}{width=\maxwidth,height=\maxheight,keepaspectratio}
% Set default figure placement to htbp
\makeatletter
\def\fps@figure{htbp}
\makeatother

\KOMAoption{captions}{tableheading}
\makeatletter
\@ifpackageloaded{tcolorbox}{}{\usepackage[many]{tcolorbox}}
\@ifpackageloaded{fontawesome5}{}{\usepackage{fontawesome5}}
\definecolor{quarto-callout-color}{HTML}{909090}
\definecolor{quarto-callout-note-color}{HTML}{0758E5}
\definecolor{quarto-callout-important-color}{HTML}{CC1914}
\definecolor{quarto-callout-warning-color}{HTML}{EB9113}
\definecolor{quarto-callout-tip-color}{HTML}{00A047}
\definecolor{quarto-callout-caution-color}{HTML}{FC5300}
\definecolor{quarto-callout-color-frame}{HTML}{acacac}
\definecolor{quarto-callout-note-color-frame}{HTML}{4582ec}
\definecolor{quarto-callout-important-color-frame}{HTML}{d9534f}
\definecolor{quarto-callout-warning-color-frame}{HTML}{f0ad4e}
\definecolor{quarto-callout-tip-color-frame}{HTML}{02b875}
\definecolor{quarto-callout-caution-color-frame}{HTML}{fd7e14}
\makeatother
\makeatletter
\makeatother
\makeatletter
\makeatother
\makeatletter
\@ifpackageloaded{caption}{}{\usepackage{caption}}
\AtBeginDocument{%
\ifdefined\contentsname
  \renewcommand*\contentsname{Table of contents}
\else
  \newcommand\contentsname{Table of contents}
\fi
\ifdefined\listfigurename
  \renewcommand*\listfigurename{List of Figures}
\else
  \newcommand\listfigurename{List of Figures}
\fi
\ifdefined\listtablename
  \renewcommand*\listtablename{List of Tables}
\else
  \newcommand\listtablename{List of Tables}
\fi
\ifdefined\figurename
  \renewcommand*\figurename{Figure}
\else
  \newcommand\figurename{Figure}
\fi
\ifdefined\tablename
  \renewcommand*\tablename{Table}
\else
  \newcommand\tablename{Table}
\fi
}
\@ifpackageloaded{float}{}{\usepackage{float}}
\floatstyle{ruled}
\@ifundefined{c@chapter}{\newfloat{codelisting}{h}{lop}}{\newfloat{codelisting}{h}{lop}[chapter]}
\floatname{codelisting}{Listing}
\newcommand*\listoflistings{\listof{codelisting}{List of Listings}}
\makeatother
\makeatletter
\@ifpackageloaded{caption}{}{\usepackage{caption}}
\@ifpackageloaded{subcaption}{}{\usepackage{subcaption}}
\makeatother
\makeatletter
\@ifpackageloaded{tcolorbox}{}{\usepackage[many]{tcolorbox}}
\makeatother
\makeatletter
\@ifundefined{shadecolor}{\definecolor{shadecolor}{rgb}{.97, .97, .97}}
\makeatother
\makeatletter
\makeatother
\ifLuaTeX
  \usepackage{selnolig}  % disable illegal ligatures
\fi
\IfFileExists{bookmark.sty}{\usepackage{bookmark}}{\usepackage{hyperref}}
\IfFileExists{xurl.sty}{\usepackage{xurl}}{} % add URL line breaks if available
\urlstyle{same} % disable monospaced font for URLs
\hypersetup{
  pdftitle={Regression (Week 17)},
  colorlinks=true,
  linkcolor={blue},
  filecolor={Maroon},
  citecolor={Blue},
  urlcolor={Blue},
  pdfcreator={LaTeX via pandoc}}

\title{Regression (Week 17)}
\author{}
\date{}

\begin{document}
\maketitle
\ifdefined\Shaded\renewenvironment{Shaded}{\begin{tcolorbox}[interior hidden, breakable, frame hidden, borderline west={3pt}{0pt}{shadecolor}, enhanced, sharp corners, boxrule=0pt]}{\end{tcolorbox}}\fi

\renewcommand*\contentsname{Table of contents}
{
\hypersetup{linkcolor=}
\setcounter{tocdepth}{3}
\tableofcontents
}
Data: RobotGLP.sav

\begin{itemize}
\tightlist
\item
  Data is available on
  \href{https://modules.lancaster.ac.uk/course/view.php?id=35502\#section-6}{Moodle}
\end{itemize}

\hypertarget{learning-objectives}{%
\section{Learning objectives}\label{learning-objectives}}

The aim of this lab is to help you to use SPSS to conduct regression
analysis, which is useful in explaining the relationship between a set
of \emph{independent} variables and a \emph{dependent variable}.

At the end of this lab, we hope that you will be able to

\begin{itemize}
\item
  Understand the meaning of independent and dependent variables
\item
  Select appropriate independent variables to explain a dependent
  variable
\item
  Produce and interpret basic SPSS outputs for multiple regression
\item
  Understand the meaning of multicollinearity, how to detect and remedy
  it
\end{itemize}

\begin{tcolorbox}[enhanced jigsaw, opacityback=0, bottomtitle=1mm, breakable, titlerule=0mm, arc=.35mm, coltitle=black, colbacktitle=quarto-callout-important-color!10!white, opacitybacktitle=0.6, colframe=quarto-callout-important-color-frame, bottomrule=.15mm, rightrule=.15mm, title=\textcolor{quarto-callout-important-color}{\faExclamation}\hspace{0.5em}{Important}, leftrule=.75mm, toprule=.15mm, toptitle=1mm, colback=white, left=2mm]

FYI, many students of the previous cohorts used regression when they
wrote their MSc dissertations.

\end{tcolorbox}

\hypertarget{why-regression}{%
\section{Why Regression}\label{why-regression}}

In marketing research, we often need to determine the impact of a set of
marketing variables on one variable -- a factor we want to understand or
predict. Furthermore, among those variables we may want to find out
which variables matter most, and which variables are not so important
that we can ignore. Regression analysis can help us findings the answers
to these questions.

In this workshop, we want to investigate factors that influence
consumers' decision to join a green loyalty program -- This LP
encourages behaviours from hotel guests that are good for environment
(e.g., reuse towels).

\begin{figure}

\includegraphics[width=0.5\textwidth,height=\textheight]{images/lab5_robot.jpg} \hfill{}

\caption{Hotel receptionist (Image courtesy of Jingxi)}

\end{figure}

In this lab, we consider the following case. A hotel manager who
recently launch a green loyalty program (GLP) wants to know about
factors that affect consumers' intention to join the hotel GLP. The
hotel manager decides to develop a survey and asks the hotel's guests to
fill in an offline survey. The survey form was handed in to hotel guests
by a frontline service robot (see the image above). The survey contains
items to measure the following constructs:

\begin{itemize}
\item
  Intention to join the green loyalty program
\item
  Anticipated guilt if not joining the green loyalty program
\item
  Perceived attractiveness of hotel receptionist
\item
  Age
\item
  Gender
\end{itemize}

\texttt{Anticipated\ guilt}, \texttt{Perceived\ Attractiveness},
\texttt{Age}, \texttt{Gender} are called the independent variables and
\texttt{Intention\ to\ join\ the\ green\ loyalty\ program} is the
dependent variable. The notion of the dependent variable comes from our
prediction that its value depends on the values of the independent
variables. The relationship between
\texttt{Intention\ to\ join\ the\ green\ LP} the and the four
independent variables can be written as:

Intention to join the green LP \textasciitilde{} Anticipated Guilt +
Perceived Attractiveness + Age + Gender + error.

You can read the above expression as a consumer' intention to join the
green loyalty program is influenced by the consumer's anticipated of
guilt, perceived attractiveness, age, and gender, and some unknown
factors represented by an error term. The relationship can be
represented by a mathematical expression as below. The names of the
variables are shortened.

\begin{equation}
Intent = \beta_0 +  \beta_1 * Guilt + \beta_2 * Attract + \beta_3 * Age + \beta_4 * Gender.
\end{equation}

where \(\beta_0\), \(\beta_1\), \(\beta_2\), \(\beta_3\), and
\(\beta_4\) are parameters that capture the impact of each of the
independent variables on \texttt{intention}.

We use regression procedure in SPSS to find out the estimates for all
parameters using sample data (\(\beta_0\) is just a constant so it is
not of our interest).

\begin{tcolorbox}[enhanced jigsaw, opacityback=0, bottomtitle=1mm, breakable, titlerule=0mm, arc=.35mm, coltitle=black, colbacktitle=quarto-callout-important-color!10!white, opacitybacktitle=0.6, colframe=quarto-callout-important-color-frame, bottomrule=.15mm, rightrule=.15mm, title=\textcolor{quarto-callout-important-color}{\faExclamation}\hspace{0.5em}{Important}, leftrule=.75mm, toprule=.15mm, toptitle=1mm, colback=white, left=2mm]

Variable \texttt{Intent}, \texttt{Guilt}, and \texttt{Attract} were
average scores of a multi-item scale where each item in the scale were
measured using a Likert scale ranging from 1 = strongly disagree to
7=strongly agree. \texttt{Age} is a continuous variable, and
\texttt{Gender} is a 0,1 variable.

\end{tcolorbox}

If you want to examine whether \texttt{Guilt} is a significant factor,
then you want to test \(H0\): \(\beta_1=0\) against \(H1\):
\(\beta_1 \ne 0\). SPSS will report the p-value associated with \(H0\).
If p-value is less than 0.05, you reject \(H0\) otherwise retain it. If
you reject \(H0\), you can conclude that \texttt{Guilt} is a significant
factor that influences \texttt{intent}.

If you hypothesize that \texttt{Guilt} is a significant factor and has a
positive impact on satisfaction, then you want to test \(H0\):
\(\beta_1=0\) against \(H1\): \(\beta_1 > 0\). SPSS will report the
p-value associated with \(H0\). You have to divide the p-value by 2. If
the p-value/2 is less than 0.05, you reject \(H0\) otherwise retain it.
If you reject \(H0\), you can conclude that \texttt{Guilt} has a
significant positive influence on \texttt{Intent}.

\hypertarget{conducting-multiple-regression-with-spss}{%
\section{Conducting Multiple Regression with
SPSS}\label{conducting-multiple-regression-with-spss}}

\textbf{Open RobotGLP.sav} -- Thanks to
\href{https://www.lancaster.ac.uk/lums/people/jingxi-huang}{Jingxi} for
allowing us to use a subset of her data to test the above model.

To conduct a regression analysis in SPSS, click the following:
\textbf{Analyze}\(\rightarrow\)\textbf{Regression}\(\rightarrow\)\textbf{Linear.}

Enter \texttt{Intent} into the dependent variable box.

Enter \texttt{Guilt},\texttt{Attract}, \texttt{Age,}Gender` into the
independent variable(s) box.

\begin{figure}

{\centering \includegraphics[width=0.7\textwidth,height=\textheight]{images/lab5_ols.png}

}

\end{figure}

Click \texttt{OK}

SPSS produces four tables. The first one does not offer much
information. Therefore, we focus our attention to the next three tables.

\begin{enumerate}
\def\labelenumi{\arabic{enumi}.}
\item
  ANOVA table: Is the model meaningful?

  \includegraphics[width=0.7\textwidth,height=\textheight]{images/lab5_anova.png}
\end{enumerate}

This table tells whether or not the model is meaningful. If the sig
value is less than 0.05, then the model is meaningful. If p-value is
greater than 0.05, then model should be dismissed and don't interpret
other outputs. In this example, the sig. value is less than 0.05;
therefore we have a meaningful model. We can proceed with the next
output.

\begin{enumerate}
\def\labelenumi{\arabic{enumi}.}
\setcounter{enumi}{1}
\tightlist
\item
  Model Summary table: How good is the model?
\end{enumerate}

Having known the model is meaningful, how do we know whether it is good
enough? The Model Summary table provides the answer to this question.

\includegraphics[width=0.7\textwidth,height=\textheight]{images/lab5_rsq.png}

The model summary table tells you how well the independent variables
explain variation in the dependent variable. The adjusted \(R^2\) is
0.311 indicating that about 31\% of the variation in the intention score
is explained by the four independent variables (Guilt, Attract, Age, and
Gender). Theoretically, the maximum possible value for adjusted R-square
is 100\% indicating a perfect model!

Adjusted \(R^2\) is \(R^2\) that is adjusted for the number of
independent variables in the model. The more independent variable you
have in the model, the larger the \(R^2\) will be. The adjusted \(R^2\)
prevents the inflation. You can use the following convention to qualify
the impact of the set of the independent variables on the dependent
variable\footnote{Ellis, Paul D. 2011. The Essential Guide to Effect
  Sizes, p.41}:

\(R^2\) ≥ 0.02 small effect

≥ 0.13 medium

≥ 0.26 large

\begin{tcolorbox}[enhanced jigsaw, opacityback=0, bottomtitle=1mm, breakable, titlerule=0mm, arc=.35mm, coltitle=black, colbacktitle=quarto-callout-note-color!10!white, opacitybacktitle=0.6, colframe=quarto-callout-note-color-frame, bottomrule=.15mm, rightrule=.15mm, title=\textcolor{quarto-callout-note-color}{\faInfo}\hspace{0.5em}{Task}, leftrule=.75mm, toprule=.15mm, toptitle=1mm, colback=white, left=2mm]

Knowing the adjusted \(R^2\), indicate the strength of the impact of
Guilt, Attract, Age, and Gender.

\end{tcolorbox}

\begin{tcolorbox}[enhanced jigsaw, opacityback=0, bottomtitle=1mm, breakable, titlerule=0mm, arc=.35mm, coltitle=black, colbacktitle=quarto-callout-warning-color!10!white, opacitybacktitle=0.6, colframe=quarto-callout-warning-color-frame, bottomrule=.15mm, rightrule=.15mm, title=\textcolor{quarto-callout-warning-color}{\faExclamationTriangle}\hspace{0.5em}{Warning}, leftrule=.75mm, toprule=.15mm, toptitle=1mm, colback=white, left=2mm]

Maximizing \(R^2\) value should not be your main goal in regression. Do
not be tempted to select variables with the aim of increasing \(R^2\)
value. \(R^2\) value in social sciences are typically in the range of
0.1 to 0.5. Selecting variables should be motivated by theories.

\end{tcolorbox}

\begin{enumerate}
\def\labelenumi{\arabic{enumi}.}
\setcounter{enumi}{2}
\tightlist
\item
  Coefficients
\end{enumerate}

\includegraphics[width=1.1\textwidth,height=\textheight]{images/lab5_coefs.png}

This table tells us which of the independent variables significantly
explain or predict the dependent variable. In this case three variables
significantly explain \texttt{Intent}. These are \texttt{Guilt},
\texttt{Attract}, and \texttt{Age}, whereas \texttt{Gender} is not
significant. Furthermore, the standardized beta values tell us which
variable has the strongest impact on \texttt{Intent}. In this case
\texttt{Attract} is the strongest predictor followed by \texttt{Guilt}
and \texttt{Age}.

\begin{tcolorbox}[enhanced jigsaw, opacityback=0, bottomtitle=1mm, breakable, titlerule=0mm, arc=.35mm, coltitle=black, colbacktitle=quarto-callout-note-color!10!white, opacitybacktitle=0.6, colframe=quarto-callout-note-color-frame, bottomrule=.15mm, rightrule=.15mm, title=\textcolor{quarto-callout-note-color}{\faInfo}\hspace{0.5em}{Task}, leftrule=.75mm, toprule=.15mm, toptitle=1mm, colback=white, left=2mm]

If you want to make a prediction, use the unstandardized coefficients
(B). If someone assigns a rating of three on \texttt{Guilt},
\texttt{Attract} can you predict her intention level? (round-off your
answer)

\end{tcolorbox}

\hypertarget{multicollinearity-problems}{%
\section{Multicollinearity Problems}\label{multicollinearity-problems}}

High correlation among independent variables is problematic in multiple
regression because it is hard for us to determine the individual
contribution of each of the independent variables in the model. For
example, if \texttt{Attract} is highly correlated with \texttt{Guilt},
then we cannot conclude that \texttt{Attract} have the strongest
influence on \texttt{Intent} because \texttt{Guilt} also gives
contribution to the magnitude of Attract-Intent relationship (it looks
stronger than it should be). In other words, the coefficients associated
with the regression estimates are biased -- not the same as their true
population values.

The situation where an independent variable is highly correlated with
another variable is referred to as multicollinearity problem. To
diagnose whether multicollinearity exists, you can check it in two ways.

\begin{enumerate}
\def\labelenumi{\arabic{enumi}.}
\tightlist
\item
  Inspect the correlation coefficient among the independent variables.
  To do correlation analysis in SPSS, follow this step:
  \textbf{Analyze}\(\rightarrow\)\textbf{Correlation}\(\rightarrow\)\textbf{Bivariate}.
  If two variables is highly correlated with the correlation coefficient
  is larger than 0.9, your model may suffer from a multicollinearity
  problem. In practice, you should be cautious, if the correlation is
  0.7.
\end{enumerate}

\begin{tcolorbox}[enhanced jigsaw, opacityback=0, bottomtitle=1mm, breakable, titlerule=0mm, arc=.35mm, coltitle=black, colbacktitle=quarto-callout-note-color!10!white, opacitybacktitle=0.6, colframe=quarto-callout-note-color-frame, bottomrule=.15mm, rightrule=.15mm, title=\textcolor{quarto-callout-note-color}{\faInfo}\hspace{0.5em}{Task}, leftrule=.75mm, toprule=.15mm, toptitle=1mm, colback=white, left=2mm]

Check the correlation coefficients among the predictors in the model.

\end{tcolorbox}

\begin{enumerate}
\def\labelenumi{\arabic{enumi}.}
\setcounter{enumi}{1}
\tightlist
\item
  Rerun your regression.
\end{enumerate}

\begin{itemize}
\item
  Click:
  \textbf{Analyze}\(\rightarrow\)\textbf{Regression}\(\rightarrow\)\textbf{Linear}.
\item
  Click \texttt{Statistics} options.
\item
  Select \texttt{Collinearity\ diagnostic}. Click \texttt{Continue} and
  Click \texttt{OK}.
\item
  Inspect the \textbf{VIF (Variance Inflation Factor} values in the
  output.
\end{itemize}

\begin{tcolorbox}[enhanced jigsaw, opacityback=0, bottomtitle=1mm, breakable, titlerule=0mm, arc=.35mm, coltitle=black, colbacktitle=quarto-callout-tip-color!10!white, opacitybacktitle=0.6, colframe=quarto-callout-tip-color-frame, bottomrule=.15mm, rightrule=.15mm, title=\textcolor{quarto-callout-tip-color}{\faLightbulb}\hspace{0.5em}{Tip}, leftrule=.75mm, toprule=.15mm, toptitle=1mm, colback=white, left=2mm]

Multicollinearity exists if VIF (variance inflation factor)
\textgreater{} 10.

\end{tcolorbox}

\includegraphics[width=1.1\textwidth,height=\textheight]{images/lab5_vif.png}

\begin{tcolorbox}[enhanced jigsaw, opacityback=0, bottomtitle=1mm, breakable, titlerule=0mm, arc=.35mm, coltitle=black, colbacktitle=quarto-callout-note-color!10!white, opacitybacktitle=0.6, colframe=quarto-callout-note-color-frame, bottomrule=.15mm, rightrule=.15mm, title=\textcolor{quarto-callout-note-color}{\faInfo}\hspace{0.5em}{Task}, leftrule=.75mm, toprule=.15mm, toptitle=1mm, colback=white, left=2mm]

See the above output, does the model suffer from a multicollinearity
problem?

\end{tcolorbox}

\hypertarget{if-multicollinearity-exists}{%
\section{If Multicollinearity
Exists}\label{if-multicollinearity-exists}}

What should you do if multicollinearity exists?

You can deal with it using one of the following techniques:

\begin{enumerate}
\def\labelenumi{\arabic{enumi}.}
\item
  Omit one or more highly correlated independent variables
\item
  Create a composite variable e.g., by taking the average score of two
  variables if the two variables causes multicollinearity
  \textbf{averaging} or using Principle Component Analysis (beyond the
  scope of our workshops)
\item
  Use the model that suffers from multicollinearity but for prediction
  purposes only.
\item
  Collect more samples
\end{enumerate}

\hypertarget{heteroskedasticity-problem}{%
\section{Heteroskedasticity Problem}\label{heteroskedasticity-problem}}

One of the assumptions in regression is that residuals or errors should
be constant across any values of independent variables. This is referred
to as the \textbf{homoskedasticity} assumption. The opposite of
homoskedasticity is \textbf{heteroskedasticity} where residuals are not
constant (i.e., heteroskedastic errors) (\emph{learn how to pronounce
these words, it took me a while to get used to them!}).

If heteroskedasticity exists, hypotheses tests about the regression
parameters are not correct anymore. See lecture slide for more details.
One of the recommended techniques to handle heteroskedasticity is to
adjust the standard errors of the regression estimates (i.e., values in
the \texttt{Std.\ error} column in the SPSS output).

\begin{tcolorbox}[enhanced jigsaw, opacityback=0, bottomtitle=1mm, breakable, titlerule=0mm, arc=.35mm, coltitle=black, colbacktitle=quarto-callout-note-color!10!white, opacitybacktitle=0.6, colframe=quarto-callout-note-color-frame, bottomrule=.15mm, rightrule=.15mm, title=\textcolor{quarto-callout-note-color}{\faInfo}\hspace{0.5em}{Task}, leftrule=.75mm, toprule=.15mm, toptitle=1mm, colback=white, left=2mm]

Rerun your regression with heteroskedasticity-adjusted standard errors
and compare the results with the original regression you had previously
conducted. Continue reading the text below.

\end{tcolorbox}

To rerun your regression, install and use the `HeteroskedasticityV3.spd'
macro developed by yours truly\footnote{Daryanto, A. (2020). Tutorial on
  heteroskedasticity using heteroskedasticityV3 SPSS macro. The
  Quantitative Methods for Psychology, 16(5), 8-20}. The macro produces
regression outputs with/without heteroscedasticity-adjusted standard
errors.

\hypertarget{sec-install}{%
\section{Installing the HeteroskedasticityV3 Macro}\label{sec-install}}

SPSS in the Lab PCs or from cloud may not permit you to install the
add-on macro because you do not have an admin right to do so. However,
it is worth trying to install the macro, it may work! Alternatively, you
need to install the macro in your personal PC where you have an admin
right to install it.

You can download the macro from these websites:

\begin{itemize}
\item
  \url{https://github.com/ahmaddaryanto/Heteroskedasticity}
\item
  Click the green button \texttt{Code}, and select
  \texttt{Download\ ZIP}.
\item
  Go to the download folder on your PC or laptop.
\item
  Locate the HeteroskedasticityV3.spd and install.
\item
  If you have an admin right, double click the file, and follow the
  instruction on your screen
\item
  If the above does not work, click
  \textbf{Extensions}\(\rightarrow\)\textbf{Utilities}\(\rightarrow\)\textbf{Install
  Custom Dialog}. Locate the file, and install.
\end{itemize}

\begin{figure}

{\centering \includegraphics{images/lab5_macro_path.png}

}

\caption{HeteroskedaticityV3 Macro in the SPSS menu after installation}

\end{figure}

\begin{figure}

{\centering \includegraphics{images/lab5_macro.png}

}

\caption{HeteroskedaticityV3 Macro}

\end{figure}

If you do not manage to install the macro. Do not panic! There is
another option without installation -- see below:

\hypertarget{running-the-macro-without-installation}{%
\section{Running the Macro without
Installation}\label{running-the-macro-without-installation}}

You can run the macro without having it installed on your machine.
Follow these steps:

\begin{itemize}
\item
  Go back to the download folder in your PC where you have downloaded
  several files from this page
  \url{https://github.com/ahmaddaryanto/Heteroskedasticity} (see
  Section~\ref{sec-install})
\item
  HeteroskedasticityV3.sps was included when you download the files from
  the github page.
\item
  Open the HeteroskedasticityV3.sps on your SPSS -- it will be opened as
  a syntax file,
\item
  Run the syntax file (highlight all the lines, run the selection
  button, i.e., the green button).
\item
  Next, open this file: \textbf{Runthemacro.sps}.
\item
  Change the DV and IVs according to your model specifications, and
\item
  Run the file.
\end{itemize}

\hypertarget{video}{%
\section{Video}\label{video}}

\href{https://dtu-panopto.lancs.ac.uk/Panopto/Pages/Viewer.aspx?id=a8f3e3c0-35c3-477c-950f-afa3007195b2}{Lecture
Week 16 on Correlation and Regression}

\href{https://dtu-panopto.lancs.ac.uk/Panopto/Pages/Viewer.aspx?id=074239e8-f79d-4ffb-8754-ac9400f4f00f}{Regression
with a categorical iv}

\href{https://dtu-panopto.lancs.ac.uk/Panopto/Pages/Viewer.aspx?id=5ba02b44-b4d1-4b70-ab45-acb900dc20a4}{Regression
diagnostics}



\end{document}
