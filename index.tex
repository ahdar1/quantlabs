% Options for packages loaded elsewhere
\PassOptionsToPackage{unicode}{hyperref}
\PassOptionsToPackage{hyphens}{url}
\PassOptionsToPackage{dvipsnames,svgnames,x11names}{xcolor}
%
\documentclass[
  letterpaper,
  DIV=11,
  numbers=noendperiod]{scrreprt}

\usepackage{amsmath,amssymb}
\usepackage{iftex}
\ifPDFTeX
  \usepackage[T1]{fontenc}
  \usepackage[utf8]{inputenc}
  \usepackage{textcomp} % provide euro and other symbols
\else % if luatex or xetex
  \usepackage{unicode-math}
  \defaultfontfeatures{Scale=MatchLowercase}
  \defaultfontfeatures[\rmfamily]{Ligatures=TeX,Scale=1}
\fi
\usepackage{lmodern}
\ifPDFTeX\else  
    % xetex/luatex font selection
\fi
% Use upquote if available, for straight quotes in verbatim environments
\IfFileExists{upquote.sty}{\usepackage{upquote}}{}
\IfFileExists{microtype.sty}{% use microtype if available
  \usepackage[]{microtype}
  \UseMicrotypeSet[protrusion]{basicmath} % disable protrusion for tt fonts
}{}
\makeatletter
\@ifundefined{KOMAClassName}{% if non-KOMA class
  \IfFileExists{parskip.sty}{%
    \usepackage{parskip}
  }{% else
    \setlength{\parindent}{0pt}
    \setlength{\parskip}{6pt plus 2pt minus 1pt}}
}{% if KOMA class
  \KOMAoptions{parskip=half}}
\makeatother
\usepackage{xcolor}
\setlength{\emergencystretch}{3em} % prevent overfull lines
\setcounter{secnumdepth}{5}
% Make \paragraph and \subparagraph free-standing
\ifx\paragraph\undefined\else
  \let\oldparagraph\paragraph
  \renewcommand{\paragraph}[1]{\oldparagraph{#1}\mbox{}}
\fi
\ifx\subparagraph\undefined\else
  \let\oldsubparagraph\subparagraph
  \renewcommand{\subparagraph}[1]{\oldsubparagraph{#1}\mbox{}}
\fi


\providecommand{\tightlist}{%
  \setlength{\itemsep}{0pt}\setlength{\parskip}{0pt}}\usepackage{longtable,booktabs,array}
\usepackage{calc} % for calculating minipage widths
% Correct order of tables after \paragraph or \subparagraph
\usepackage{etoolbox}
\makeatletter
\patchcmd\longtable{\par}{\if@noskipsec\mbox{}\fi\par}{}{}
\makeatother
% Allow footnotes in longtable head/foot
\IfFileExists{footnotehyper.sty}{\usepackage{footnotehyper}}{\usepackage{footnote}}
\makesavenoteenv{longtable}
\usepackage{graphicx}
\makeatletter
\def\maxwidth{\ifdim\Gin@nat@width>\linewidth\linewidth\else\Gin@nat@width\fi}
\def\maxheight{\ifdim\Gin@nat@height>\textheight\textheight\else\Gin@nat@height\fi}
\makeatother
% Scale images if necessary, so that they will not overflow the page
% margins by default, and it is still possible to overwrite the defaults
% using explicit options in \includegraphics[width, height, ...]{}
\setkeys{Gin}{width=\maxwidth,height=\maxheight,keepaspectratio}
% Set default figure placement to htbp
\makeatletter
\def\fps@figure{htbp}
\makeatother

\KOMAoption{captions}{tableheading}
\makeatletter
\@ifpackageloaded{tcolorbox}{}{\usepackage[skins,breakable]{tcolorbox}}
\@ifpackageloaded{fontawesome5}{}{\usepackage{fontawesome5}}
\definecolor{quarto-callout-color}{HTML}{909090}
\definecolor{quarto-callout-note-color}{HTML}{0758E5}
\definecolor{quarto-callout-important-color}{HTML}{CC1914}
\definecolor{quarto-callout-warning-color}{HTML}{EB9113}
\definecolor{quarto-callout-tip-color}{HTML}{00A047}
\definecolor{quarto-callout-caution-color}{HTML}{FC5300}
\definecolor{quarto-callout-color-frame}{HTML}{acacac}
\definecolor{quarto-callout-note-color-frame}{HTML}{4582ec}
\definecolor{quarto-callout-important-color-frame}{HTML}{d9534f}
\definecolor{quarto-callout-warning-color-frame}{HTML}{f0ad4e}
\definecolor{quarto-callout-tip-color-frame}{HTML}{02b875}
\definecolor{quarto-callout-caution-color-frame}{HTML}{fd7e14}
\makeatother
\makeatletter
\makeatother
\makeatletter
\@ifpackageloaded{bookmark}{}{\usepackage{bookmark}}
\makeatother
\makeatletter
\@ifpackageloaded{caption}{}{\usepackage{caption}}
\AtBeginDocument{%
\ifdefined\contentsname
  \renewcommand*\contentsname{Table of contents}
\else
  \newcommand\contentsname{Table of contents}
\fi
\ifdefined\listfigurename
  \renewcommand*\listfigurename{List of Figures}
\else
  \newcommand\listfigurename{List of Figures}
\fi
\ifdefined\listtablename
  \renewcommand*\listtablename{List of Tables}
\else
  \newcommand\listtablename{List of Tables}
\fi
\ifdefined\figurename
  \renewcommand*\figurename{Figure}
\else
  \newcommand\figurename{Figure}
\fi
\ifdefined\tablename
  \renewcommand*\tablename{Table}
\else
  \newcommand\tablename{Table}
\fi
}
\@ifpackageloaded{float}{}{\usepackage{float}}
\floatstyle{ruled}
\@ifundefined{c@chapter}{\newfloat{codelisting}{h}{lop}}{\newfloat{codelisting}{h}{lop}[chapter]}
\floatname{codelisting}{Listing}
\newcommand*\listoflistings{\listof{codelisting}{List of Listings}}
\makeatother
\makeatletter
\@ifpackageloaded{caption}{}{\usepackage{caption}}
\@ifpackageloaded{subcaption}{}{\usepackage{subcaption}}
\makeatother
\makeatletter
\@ifpackageloaded{tcolorbox}{}{\usepackage[skins,breakable]{tcolorbox}}
\makeatother
\makeatletter
\@ifundefined{shadecolor}{\definecolor{shadecolor}{rgb}{.97, .97, .97}}
\makeatother
\makeatletter
\makeatother
\makeatletter
\makeatother
\ifLuaTeX
  \usepackage{selnolig}  % disable illegal ligatures
\fi
\IfFileExists{bookmark.sty}{\usepackage{bookmark}}{\usepackage{hyperref}}
\IfFileExists{xurl.sty}{\usepackage{xurl}}{} % add URL line breaks if available
\urlstyle{same} % disable monospaced font for URLs
\hypersetup{
  pdftitle={MKTG402(3) Labs},
  pdfauthor={Ahmad Daryanto},
  colorlinks=true,
  linkcolor={blue},
  filecolor={Maroon},
  citecolor={Blue},
  urlcolor={Blue},
  pdfcreator={LaTeX via pandoc}}

\title{MKTG402(3) Labs}
\author{Ahmad Daryanto}
\date{February 13, 2024}

\begin{document}
\maketitle
\ifdefined\Shaded\renewenvironment{Shaded}{\begin{tcolorbox}[interior hidden, boxrule=0pt, frame hidden, borderline west={3pt}{0pt}{shadecolor}, enhanced, sharp corners, breakable]}{\end{tcolorbox}}\fi

\renewcommand*\contentsname{Table of contents}
{
\hypersetup{linkcolor=}
\setcounter{tocdepth}{2}
\tableofcontents
}
\bookmarksetup{startatroot}

\hypertarget{preface}{%
\chapter*{Preface}\label{preface}}
\addcontentsline{toc}{chapter}{Preface}

\markboth{Preface}{Preface}

This is a website for the workshops of MKTG402(3) - Introduction to
Quantitative Research Methods for students at the Advanced Marketing
Management (AMM) program at Lancaster University.

Instructor:
\href{https://sites.google.com/view/ahmaddaryanto}{\textbf{Dr.~Ahmad
Daryanto}}

\hypertarget{document-version}{%
\section*{Document Version}\label{document-version}}
\addcontentsline{toc}{section}{Document Version}

\markright{Document Version}

The pdf version of this document is available to download from Moodle.

\hypertarget{spss-version}{%
\section*{SPSS Version}\label{spss-version}}
\addcontentsline{toc}{section}{SPSS Version}

\markright{SPSS Version}

I am using SPSS version 28. You can use any SPSS versions (e.g., ver 26,
27, and 29), where the statistical procedures used in the labs are the
same but the output formats are slightly different from those of version
28. Therefore, I recommend you to use version 28, so you can check
whether your outputs are the same with those shown in this document.

\hypertarget{textbook}{%
\section*{Textbook}\label{textbook}}
\addcontentsline{toc}{section}{Textbook}

\markright{Textbook}

I do not use a specific textbook. Any marketing research or quantitative
research textbook can be used for your self-learning, in particular, I
recommend the following text:

\begin{itemize}
\item
  Malhotra, N.K, Birks, D., and Wills, P. (2012) Marketing Research: An
  applied orientation, 4\textsuperscript{th} edition, London:
  Prentice-Hall, Pearson. Other books on marketing research by Malhotra
  are also useful.
\item
  Sarstedt, Marko and Erik Mooi (2014). A Concise Guide to Market
  Research. Springer. As the title says, this book is concise focusing
  statistical tests typically conducted in market research. The
  organization of the chapters in the book match closely with what we
  cover in this module. You can read the e-version of the first edition
  via our library website.
\item
  Feinberg, Kinnear, and Taylor. (2013). Modern Marketing Research:
  Concept, Methods, and Cases, 2\textsuperscript{nd} edition, Cengage.
  Written by excellent researchers in Marketing, I also like this book.
  It has plenty of mini cases that help readers to see the application
  of concepts in practices.
\end{itemize}

\begin{center}\rule{0.5\linewidth}{0.5pt}\end{center}

\hypertarget{overview}{%
\section*{Overview}\label{overview}}
\addcontentsline{toc}{section}{Overview}

\markright{Overview}

\hypertarget{workshop-1-week-13-getting-started-with-spss}{%
\subsection*{\texorpdfstring{\protect\hyperlink{sec-lab1}{Workshop 1/
Week 13: Getting started with
SPSS}}{Workshop 1/ Week 13: Getting started with SPSS}}\label{workshop-1-week-13-getting-started-with-spss}}
\addcontentsline{toc}{subsection}{\protect\hyperlink{sec-lab1}{Workshop
1/ Week 13: Getting started with SPSS}}

Tasks: In this workshop, we will label variables, create new ones, and
code scales.

\hypertarget{workshop-2-week-14-descriptive-statistics-graphs-and-charts-in-spss}{%
\subsection*{\texorpdfstring{\protect\hyperlink{sec-lab2}{Workshop 2/
Week 14: Descriptive statistics, graphs and charts in
SPSS}}{Workshop 2/ Week 14: Descriptive statistics, graphs and charts in SPSS}}\label{workshop-2-week-14-descriptive-statistics-graphs-and-charts-in-spss}}
\addcontentsline{toc}{subsection}{\protect\hyperlink{sec-lab2}{Workshop
2/ Week 14: Descriptive statistics, graphs and charts in SPSS}}

Tasks: In this workshop, we will use SPSS to produce descriptive
statistics, graphs and charts for different types of variables.

\hypertarget{workshop-3-week-15-crosstabs-and-two-independent-samples-t-test}{%
\subsection*{\texorpdfstring{\protect\hyperlink{sec-lab3}{Workshop 3/
Week 15: Crosstabs and two independent samples
t-test}}{Workshop 3/ Week 15: Crosstabs and two independent samples t-test}}\label{workshop-3-week-15-crosstabs-and-two-independent-samples-t-test}}
\addcontentsline{toc}{subsection}{\protect\hyperlink{sec-lab3}{Workshop
3/ Week 15: Crosstabs and two independent samples t-test}}

Tasks: This is the first of the bivariate analysis workshops. Here we
will look at the analysis of 2 sets of nominal data in a cross
tabulation and use a Chi-square test to determine whether a significant
difference between the groups has occurred. In the second part of the
workshop, we will look at the analysis of mean scores, i.e., comparing
mean scores of two groups with two independent samples T-test.

\hypertarget{workshop-4-week-16-anova-and-experimentation}{%
\subsection*{\texorpdfstring{\protect\hyperlink{sec-lab4}{Workshop 4/
Week 16: ANOVA and
Experimentation}}{Workshop 4/ Week 16: ANOVA and Experimentation}}\label{workshop-4-week-16-anova-and-experimentation}}
\addcontentsline{toc}{subsection}{\protect\hyperlink{sec-lab4}{Workshop
4/ Week 16: ANOVA and Experimentation}}

Tasks: Here we will use SPSS to compare the mean scores of two or more
nominal groups. Using one-way ANOVA, we will determine if significant
differences exists between two groups or more. Then we will look two-way
ANOVA, specifically a 2x3 ANOVA between-subject design.

\hypertarget{workshop-5-week-17-regression}{%
\subsection*{\texorpdfstring{\protect\hyperlink{sec-lab5}{Workshop 5/
Week 17:
Regression}}{Workshop 5/ Week 17: Regression}}\label{workshop-5-week-17-regression}}
\addcontentsline{toc}{subsection}{\protect\hyperlink{sec-lab5}{Workshop
5/ Week 17: Regression}}

Tasks: Using correlation we will determine the nature of the association
between two sets of interval or ratio data. Then using linear
regressions, we will develop a model to examine the influence of a set
of independent variables on one dependent variable.

\newpage{}

\bookmarksetup{startatroot}

\hypertarget{sec-lab1}{%
\chapter{Getting started with SPSS (Week 13)}\label{sec-lab1}}

Data: Dell.sav

\begin{itemize}
\item
  Data is available on Moodle
\item
  On Moodle, go to a section \texttt{Workshop\ Materials}, and open a
  folder \texttt{Workshop\ 1,\ 2,\ 3\ -\ Data\ and\ Articles}. Download
  Dell.sav and \textbf{Dell Questionnaire} .
\end{itemize}

\hypertarget{learning-objectives}{%
\section{Learning objectives}\label{learning-objectives}}

The aim of this first lab exercise is to get you familiar with the SPSS
program.

Learning objectives:

At the end of this lab, we hope that you will be able to:

\begin{itemize}
\item
  Explore a data set and produce descriptive statistics in table format
\item
  Transfer tables into Word
\end{itemize}

In this lab we will use a data set from real customers that is available
to download from
\href{https://modules.lancaster.ac.uk/mod/folder/view.php?id=1959818}{Moodle}.
The data set contains information from 372 customers collected from a
survey of purchasers of Dell PCs and notebooks. With this survey, Dell
wants to understand their customers' primary usage for internet and
their customers' satisfaction with their purchases. Customers also give
information about their demographics such as age and gender. Your
primary job today is to describe the characteristics of the sample.

\hypertarget{questionnaire-and-data}{%
\section{Questionnaire and Data}\label{questionnaire-and-data}}

Before opening that data and starting to analyse it, it is very
important that you understand what was measured, how it was measured and
as a result what level of measurement has been used. As we go on in the
course it is very important to understand the distinctions between
nominal, ordinal, interval, and ratio data.

\begin{tcolorbox}[enhanced jigsaw, bottomtitle=1mm, toptitle=1mm, arc=.35mm, left=2mm, colframe=quarto-callout-note-color-frame, rightrule=.15mm, breakable, opacityback=0, coltitle=black, colback=white, colbacktitle=quarto-callout-note-color!10!white, opacitybacktitle=0.6, toprule=.15mm, titlerule=0mm, title=\textcolor{quarto-callout-note-color}{\faInfo}\hspace{0.5em}{Task}, bottomrule=.15mm, leftrule=.75mm]

Have a look at the questionnaire and identify which questions use a
nominal, interval, and ratio scale.

\end{tcolorbox}

\hypertarget{working-with-spss}{%
\section{Working with SPSS}\label{working-with-spss}}

Open the data file Dell.sav, which you must download from Moodle. You
should save the data to your own drive and then double-click the file
name from windows explorer.

There are two different views of the data which can be seen by clicking
the bottom left tab. The data view shows the imputed data for each
respondent (each row represents a respondent and each column represents
a variable or questionnaire item). The variable view shows the detail of
what has been measured. The key columns are the Name, Labels and Values
columns which give the shortened variable name (e.g.~see variable
\texttt{q2\_job} in the third row), the full details of what has been
measured as a label (e.g.~please indicate which of these you have ever
done on the Internet: Looked for a job) and the Values represent what
the imputed numbers represented in the questionnaire (e.g.~0 = ``don't
know'', 1 = ``looked for a job'', 2= don't look for a job).

\begin{figure}

{\centering \includegraphics{images/lab1_dataview.png}

}

\caption{Data View: Now click variable view at the bottom of the
window.}

\end{figure}

\begin{figure}

{\centering \includegraphics{images/lab1_varview.png}

}

\caption{Variable View}

\end{figure}

\hypertarget{creating-tables}{%
\section{Creating Tables}\label{creating-tables}}

I want you to try and use SPSS on your own (I will come around and
answer questions that you have). You can explore the data set as much as
you wish but here are some ideas of tables which might be interesting to
explore.

\hypertarget{frequency-tables}{%
\section{Frequency Tables}\label{frequency-tables}}

This table allows you to explore only one variable in each table. Use
the commands \textbf{Analyze}\(\rightarrow\)\textbf{Descriptive
statistics}\(\rightarrow\)\textbf{Frequencies}. If you click on the
\texttt{Charts} tab you can also get SPSS to generate bar, pie, and
histograms for the variables selected.

Note that you can get many frequency tables in the one command by
selecting multiple variables from the list of those available. To select
more than one variable at a time keep the \texttt{Ctrl} button down. The
frequency tables will be displayed in the output window.

\begin{tcolorbox}[enhanced jigsaw, bottomtitle=1mm, toptitle=1mm, arc=.35mm, left=2mm, colframe=quarto-callout-note-color-frame, rightrule=.15mm, breakable, opacityback=0, coltitle=black, colback=white, colbacktitle=quarto-callout-note-color!10!white, opacitybacktitle=0.6, toprule=.15mm, titlerule=0mm, title=\textcolor{quarto-callout-note-color}{\faInfo}\hspace{0.5em}{Task}, bottomrule=.15mm, leftrule=.75mm]

Why not create a frequency table for the variables named
\texttt{q2\_job}, \texttt{q2\_trip}?

\end{tcolorbox}

\begin{figure}

{\centering \includegraphics{images/lab1_dataview_with_menu.png}

}

\caption{Data view with menu options displayed}

\end{figure}

\begin{figure}

{\centering \includegraphics{images/lab1_output.png}

}

\caption{Outputs}

\end{figure}

\begin{tcolorbox}[enhanced jigsaw, bottomtitle=1mm, toptitle=1mm, arc=.35mm, left=2mm, colframe=quarto-callout-note-color-frame, rightrule=.15mm, breakable, opacityback=0, coltitle=black, colback=white, colbacktitle=quarto-callout-note-color!10!white, opacitybacktitle=0.6, toprule=.15mm, titlerule=0mm, title=\textcolor{quarto-callout-note-color}{\faInfo}\hspace{0.5em}{Task}, bottomrule=.15mm, leftrule=.75mm]

\textbf{Now, I want you to do more analysis to be able to answer the
following questions:}

1. What can you say about the sample regarding the distribution of the
number of hours that customers spend online? (Q1)

2. What are the proportions of male and female in the sample?
(\texttt{Q14\_gender})

3. What is the percentage of Dell customers in the sample who are
willing to joint Dell loyalty program? (\texttt{Q15\_loyal})

\end{tcolorbox}

\hypertarget{getting-help-in-spss}{%
\section{Getting Help in SPSS}\label{getting-help-in-spss}}

If you want to find out more about interpreting the frequency table, why
not try using the \texttt{Help} tab? Select again
\textbf{Analyze}\(\rightarrow\) \textbf{Descriptive statistics}
\(\rightarrow\) \textbf{Frequencies}, and click on the \texttt{Help} tab
at the bottom. You will find that a new window opens up and provides
useful information. SPSS also has a range of support features which can
be found using the \texttt{Help} menu. Spend some time looking through
the options of particular interest maybe the Statistics Coach, Tutorials
and Case Studies -- you can click this link
\href{https://www.ibm.com/docs/en/spss-statistics/28.0.0?topic=SSLVMB_28.0.0/statistics_mainhelp_ddita/spss/base/overvw_auto_0.html}{Getting
help - IBM documentation}, which directly take you to the online help
page of SPSS ver. 28.

\hypertarget{cross-tabulations}{%
\section{Cross Tabulations}\label{cross-tabulations}}

What are the proportions of male vs.~female that are willing to join
Dell loyalty program? To answer this question, you need to do cross
tabulations. Use the commands
\textbf{Analyze}\(\rightarrow\)\textbf{Descriptive
statistics}\(\rightarrow\)\textbf{Crosstabs}.

Crosstabs are used to explore the relationship between two categorical
variables (e.g., nominal by nominal, nominal vs.~ordinal). Use the cells
tab to select row or column percentages in order to be able to compare
across the groups. The type of percentage will depend on what basis you
want to compare i.e.~whether the rows or the columns should add to
100\%. Note that you can also produce graphs (by checking the
\texttt{Display\ clustered\ bar\ charts} option).

\begin{tcolorbox}[enhanced jigsaw, bottomtitle=1mm, toptitle=1mm, arc=.35mm, left=2mm, colframe=quarto-callout-note-color-frame, rightrule=.15mm, breakable, opacityback=0, coltitle=black, colback=white, colbacktitle=quarto-callout-note-color!10!white, opacitybacktitle=0.6, toprule=.15mm, titlerule=0mm, title=\textcolor{quarto-callout-note-color}{\faInfo}\hspace{0.5em}{Task}, bottomrule=.15mm, leftrule=.75mm]

What about exploring the relationship between gender and level of
education?

\end{tcolorbox}

\hypertarget{copying-spss-output-into-word}{%
\section{Copying SPSS output into
word}\label{copying-spss-output-into-word}}

For reports or as a record of what analysis you have done you may want
to put your charts and tables into an MS Word document. This is straight
forward, all you need to do is position the cursors on the SPSS output
(chart or table) and right click, select \texttt{copy} and paste the
table or graph into your MS Word document.

\hypertarget{codebook}{%
\section{Codebook}\label{codebook}}

If you use a paper-and-pencil survey, you need to enter the data
manually. Before you enter the data into an SPSS data view window, I
recommend that you create a codebook in advance. A codebookt contains
names and descriptions of variables and coding for response answers
including missing values, which is built directly from your
questionnaire. A codebook is very useful and handy to place on your desk
when you are working on your data where you quickly check the variables
that you use, and communicate the SPSS outputs to your team members.

I also recommend that you also create a codebook when you use an online
survey (e.g.,
\href{https://lancasteruni.eu.qualtrics.com/homepage/ui}{https://lancasteruni.eu.qualtrics.com}.
Online surveys like qualtrics.com enable you to import data into an SPSS
format (.sav). However, you may need to clean this data before you can
use it (you may not need to do so if you want a quick report). For
instance, the column label in the SPSS variable view window will be
written by the software automatically. When you have a codebook, you can
leave this column as blank and adjust the variable names as the way they
are written in your codebook.

Example of a codebook

\begin{longtable}[]{@{}
  >{\raggedright\arraybackslash}p{(\columnwidth - 8\tabcolsep) * \real{0.0531}}
  >{\raggedright\arraybackslash}p{(\columnwidth - 8\tabcolsep) * \real{0.1372}}
  >{\raggedright\arraybackslash}p{(\columnwidth - 8\tabcolsep) * \real{0.5487}}
  >{\raggedright\arraybackslash}p{(\columnwidth - 8\tabcolsep) * \real{0.1239}}
  >{\raggedright\arraybackslash}p{(\columnwidth - 8\tabcolsep) * \real{0.1239}}@{}}
\toprule\noalign{}
\begin{minipage}[b]{\linewidth}\raggedright
Variable
\end{minipage} & \begin{minipage}[b]{\linewidth}\raggedright
Construct
\end{minipage} & \begin{minipage}[b]{\linewidth}\raggedright
Statement
\end{minipage} & \begin{minipage}[b]{\linewidth}\raggedright
Response
\end{minipage} & \begin{minipage}[b]{\linewidth}\raggedright
Source
\end{minipage} \\
\midrule\noalign{}
\endhead
\bottomrule\noalign{}
\endlastfoot
q1\_online & Amount of hours spent online & Approximately how many hours
per week do you spend online? (This would be the total from all the
locations you might use) & 1= Less than 1 hour

2=1-5 hours

3=6-10 hours

4=11-20 hours

5=21-40 hours

6=41 hours more

99=Missing values & Self-developed \\
\ldots{} & \ldots{} & \ldots{} & \ldots{} & \\
q4\_sat & Satisfaction Overall & How satisfied are you with your DELL
computer system? & 1= Very dissatisfied

2 = Somewhat dissatisfied

3 = Somewhat satisfied

4 = Very satisfied

99 = Missing values & Adapted from Oliver(1993) \\
& & & & \\
\end{longtable}

\hypertarget{lancaster-qualtrics}{%
\section{Lancaster Qualtrics}\label{lancaster-qualtrics}}

Lancaster University has a subscription to an online survey Qualtrics.
You can use your university email to log onto Qualtrics.

\begin{tcolorbox}[enhanced jigsaw, bottomtitle=1mm, toptitle=1mm, arc=.35mm, left=2mm, colframe=quarto-callout-note-color-frame, rightrule=.15mm, breakable, opacityback=0, coltitle=black, colback=white, colbacktitle=quarto-callout-note-color!10!white, opacitybacktitle=0.6, toprule=.15mm, titlerule=0mm, title=\textcolor{quarto-callout-note-color}{\faInfo}\hspace{0.5em}{Task}, bottomrule=.15mm, leftrule=.75mm]

Click this link
\href{https://lancasteruni.eu.qualtrics.com/homepage/ui}{https://lancasteruni.eu.qualtrics.com},
and log onto Qualtrics.

\end{tcolorbox}

\begin{tcolorbox}[enhanced jigsaw, bottomtitle=1mm, toptitle=1mm, arc=.35mm, left=2mm, colframe=quarto-callout-important-color-frame, rightrule=.15mm, breakable, opacityback=0, coltitle=black, colback=white, colbacktitle=quarto-callout-important-color!10!white, opacitybacktitle=0.6, toprule=.15mm, titlerule=0mm, title=\textcolor{quarto-callout-important-color}{\faExclamation}\hspace{0.5em}{Important}, bottomrule=.15mm, leftrule=.75mm]

Find more information about Qualtrics (e.g., training, how to create and
edit a survey, etc) at the ISS page
\href{https://answers.lancaster.ac.uk/display/ISS/Qualtrics+access\%2C+help+and+training}{here}

\end{tcolorbox}

\newpage{}

\bookmarksetup{startatroot}

\hypertarget{sec-lab2}{%
\chapter{Descriptive Statistics, Graphs and Charts in SPSS (Week
14)}\label{sec-lab2}}

Data: Dell.sav

Data is available on `Workshop Materials' folder on Moodle.

\hypertarget{learning-objectives-1}{%
\section{Learning objectives}\label{learning-objectives-1}}

The aim of this second lab exercise is to get you familiar with the SPSS
program in terms of further descriptive statistics and basic
computations. Through completing this lab you will be able to:

\begin{itemize}
\tightlist
\item
  Explore a data set and produce descriptive statistics in graph format.
\item
  Produce descriptive statistic output.
\item
  Compute a new variable based on the data provided.
\end{itemize}

Last time we focused on producing tables, in this lab we are now
focusing on producing charts and instead of using the \texttt{Analyze}
menu we will be using the \texttt{Graphs} menu.

\hypertarget{charts}{%
\section{Charts}\label{charts}}

You can use the charts from the commands already undertaken in lab 1,
however if you want to explore other charts or customize a chart you
should use the specifically developed chart tool. If you use SPSS ver 28
and below, click \textbf{Graphs}\(\rightarrow\)\textbf{Legacy dialogs}
and then select the type of chart that you want. In SPSS ver 29, SPSS
gets rid of \textbf{Legacy dialogs} -- you will see list of graphs you
can use after clicking \textbf{Graphs}.

To start off with, try a pie chart for gender (\texttt{q14\_gender}), a
bar chart for level of education (\texttt{q11\_grade}) and a box plot
for satisfaction (\texttt{q4\_sat}).

Make sure that you use appropriate titles for your charts and label the
axes accordingly. Try exploring counts and percentages as well as the
different chart types.

\begin{tcolorbox}[enhanced jigsaw, bottomtitle=1mm, toptitle=1mm, arc=.35mm, left=2mm, colframe=quarto-callout-warning-color-frame, rightrule=.15mm, breakable, opacityback=0, coltitle=black, colback=white, colbacktitle=quarto-callout-warning-color!10!white, opacitybacktitle=0.6, toprule=.15mm, titlerule=0mm, title=\textcolor{quarto-callout-warning-color}{\faExclamationTriangle}\hspace{0.5em}{Attention}, bottomrule=.15mm, leftrule=.75mm]

When each dialogue box appears you need to select the correct summary.
For the bar chart you want to select a simple bar chart with ``summaries
for groups of cases'', you also want ``summaries for groups of cases''
for the pie chart but for the box plot you should select ``summaries for
separate variables''.

\end{tcolorbox}

If you want to explore putting data labels on your charts and for
instance changing the color, in the output widow double click on the
chart and it will open up in an editor view as the following screen shot
shows

\begin{figure}

{\centering \includegraphics[width=5.04167in,height=\textheight]{images/lab2_pie.png}

}

\caption{Pie chart}

\end{figure}

\begin{tcolorbox}[enhanced jigsaw, bottomtitle=1mm, toptitle=1mm, arc=.35mm, left=2mm, colframe=quarto-callout-note-color-frame, rightrule=.15mm, breakable, opacityback=0, coltitle=black, colback=white, colbacktitle=quarto-callout-note-color!10!white, opacitybacktitle=0.6, toprule=.15mm, titlerule=0mm, title=\textcolor{quarto-callout-note-color}{\faInfo}\hspace{0.5em}{Task}, bottomrule=.15mm, leftrule=.75mm]

Which type of chart is best suited to the following types of data:
nominal, ordinal, interval or ratio?

\end{tcolorbox}

\hypertarget{descriptive-statistics}{%
\section{Descriptive Statistics}\label{descriptive-statistics}}

Let's now try using the Descriptives function using the commands
\textbf{Analyze}\(\rightarrow\)\textbf{Descriptive
statistics}\(\rightarrow\)\textbf{Descriptives}. Here you can calculate
mean, standard deviations and other measures of central tendency and
dispersion

\begin{tcolorbox}[enhanced jigsaw, bottomtitle=1mm, toptitle=1mm, arc=.35mm, left=2mm, colframe=quarto-callout-note-color-frame, rightrule=.15mm, breakable, opacityback=0, coltitle=black, colback=white, colbacktitle=quarto-callout-note-color!10!white, opacitybacktitle=0.6, toprule=.15mm, titlerule=0mm, title=\textcolor{quarto-callout-note-color}{\faInfo}\hspace{0.5em}{Task}, bottomrule=.15mm, leftrule=.75mm]

Why not try and explore the satisfaction of DELL customers
(\texttt{q4\_sat}) and also the amount that customers have spent on
internet in the 12 month (\texttt{q16\_spent})?

\end{tcolorbox}

\hypertarget{computing-new-variables}{%
\section{Computing New Variables}\label{computing-new-variables}}

It is also important to be able to perform basic manipulations of the
data. The most important manipulation is the creation of new variables.
Referring back to the week 2 lecture you will remember that a lot of
marketing research data is collected through the use of multi-item
scales. The data from the multi-item scale is then averaged into
\textbf{a single composite measure}. Most of the DELL questionnaire is
developed from scales used in marketing literature. You should practice
creating composite variables for concepts such as Market Maven, Opinion
Leadership, and Innovativeness. You can identify which items capture
which concept from the variable name which is best seen from the
variable view which you can click to from the bottom left hand corner of
the screen.

\hypertarget{reliability-analysis---computing-cronbachs-alpha}{%
\section{Reliability Analysis - Computing Cronbach's
alpha}\label{reliability-analysis---computing-cronbachs-alpha}}

Before creating a new composite variable it is important to check that
the items exhibit internal consistency reliability (also discussed in
the second lecture). You can do this by selecting the menu options
\textbf{Analyze}\(\rightarrow\)
\textbf{Scale}\(\rightarrow\)\textbf{Reliability Analysis}. You then
select each of the items used to measure the construct.

Conduct a reliability analysis for Opinion Leadership (Variable names:
\texttt{q10\_op1}, \texttt{Q10\_op2}, \texttt{Q10\_op3}). If the alpha
value given in the output is 0.7 or above then you can create a new
composite measure by averaging the items together, knowing that all the
items get well as a group.

If the alpha is below 0.7 it may be because some items in the scale are
reversed coded, or do not get well with the rest. In order to find out
which scale items need to be reversed or removed, when the alpha dialog
box is open click the \texttt{statistics} tab and then check
\texttt{Scale\ if\ item\ deleted}.

The alpha value for Opinion Leadership is 0.927 and therefore the three
items can be averaged. To do this use the menu commands
\textbf{Transform}\(\rightarrow\)\textbf{Compute}. The new variable will
appear at the end of the dataset so please look for it in the data view.
Note that you will be able to name the new variable whatever you like
but I would suggest calling it \texttt{Opinion\_avg}.

\begin{figure}

{\centering \includegraphics[width=6.09375in,height=\textheight]{images/lab2_opinion.png}

}

\caption{Computing mean}

\end{figure}

\begin{tcolorbox}[enhanced jigsaw, bottomtitle=1mm, toptitle=1mm, arc=.35mm, left=2mm, colframe=quarto-callout-note-color-frame, rightrule=.15mm, breakable, opacityback=0, coltitle=black, colback=white, colbacktitle=quarto-callout-note-color!10!white, opacitybacktitle=0.6, toprule=.15mm, titlerule=0mm, title=\textcolor{quarto-callout-note-color}{\faInfo}\hspace{0.5em}{Task}, bottomrule=.15mm, leftrule=.75mm]

Why not try and explore the new variable you just created? (e.g., report
mean, standard deviation, inspect the shape of its distribution, create
a boxplot)

\end{tcolorbox}

\hypertarget{video}{%
\section{Video}\label{video}}

\href{https://dtu-panopto.lancs.ac.uk/Panopto/Pages/Viewer.aspx?id=e8b97ce8-7b60-4344-8a42-ac8c00fa05fc}{Boxplot}

\href{https://dtu-panopto.lancs.ac.uk/Panopto/Pages/Viewer.aspx?id=0213e35a-95a5-46da-a39b-ac8c01036a34}{Reliability
analysis}

\hypertarget{readings}{%
\section{Readings}\label{readings}}

Feick, L. F., \& Price, L. L. (1987). The market maven: A diffuser of
marketplace information. \emph{Journal of Marketing}, 51(1), 83-97.

Goldsmith, R. E., Flynn, L. R., \& Goldsmith, E. B. (2003). Innovative
consumers and market mavens. \emph{Journal of Marketing Theory and
Practice}, 11(4), 54-65.

\bookmarksetup{startatroot}

\hypertarget{sec-lab3}{%
\chapter{Crosstabs and Two Independent Samples T-test (Week
15)}\label{sec-lab3}}

Data: Dell.sav

\begin{itemize}
\tightlist
\item
  Data is available on
  \href{https://modules.lancaster.ac.uk/course/view.php?id=35502\#section-6}{Moodle}
\end{itemize}

\hypertarget{learning-objectives-2}{%
\section{Learning objectives}\label{learning-objectives-2}}

The aim of this third lab is to help you to use SPSS to examine group
differences based on demographic factors. In this lab we will cover:

\begin{itemize}
\item
  Chi-square test
\item
  The two independent samples t test
\end{itemize}

\hypertarget{chi-square-test}{%
\section{Chi-square Test}\label{chi-square-test}}

The chi-square test is useful for determining if differences exist
between two categorical variables. This test can be used to substantiate
perceived associations when calculating crosstabs such as those that you
did in lab 1. Let's say that you want to explore if there is an
association between gender (\texttt{Q14\_gender}) and number of hours
spent on internet (\texttt{Q1\_online}). A chi-square test would be
useful to assess this.

As in lab 1, use the menu options
\textbf{Analyze}\(\rightarrow\)\textbf{Descriptive
statistics}\(\rightarrow\)\textbf{Crosstabs}. I prefer to insert
\texttt{q14\_gender} into column and \texttt{q1\_online} into row
window--this way, you will see the distribution of gender within each
category of \texttt{q1\_online}. Select the statistics tab and click on
the Chi-square option. Note that you can also click on
\texttt{Phi\ and\ Cramer’s\ V\ to} get an indication of the strength of
the association.

Using the cells tab you should also select row or column percents to
help you to see the pattern of association (in this example, I prefer to
select row percentages)

In the output window you will see that three (or four with the addition
Phi and Cramer's V) tables are produced. You can ignore the
\texttt{Case\ Processing\ Summary}.

The second table is the crosstab table that you produced in lab 1 and
the third table \texttt{Chi-Square\ Tests} gives you the new results.
You will see that you have a table and graph like the one below for the
gender by online crosstabs.

\begin{figure}

{\centering \includegraphics[width=0.8\textwidth,height=\textheight]{images/lab3_pic1_revised.png}

}

\caption{Frequency Table}

\end{figure}

\begin{figure}

{\centering \includegraphics[width=0.8\textwidth,height=\textheight]{images/lab3_pic2.png}

}

\caption{Clustered Bar Chart}

\end{figure}

The p-value associated with the crosstabs is given by the following
table

\begin{figure}

{\centering \includegraphics[width=0.8\textwidth,height=\textheight]{images/lab3_pic3.png}

}

\caption{SPSS Output: Chi-Square Tests}

\end{figure}

The most important column within the table is the ``Asymp. Sig.
(2-sided)''. This is the p-value column and the result above indicates
that there is an evidence of an association between the rows and columns
(because the p-value is smaller than 0.05), in this case between gender
of the consumer and number of hours online. It is important to look at
the proportion of consumers in each of the six groups.

If there was a statistically significant association you would ask the
following:

\begin{enumerate}
\def\labelenumi{\arabic{enumi}.}
\item
  Where do you see the association between the proportions of consumers?
\item
  How strong is the association?
\end{enumerate}

\begin{tcolorbox}[enhanced jigsaw, bottomtitle=1mm, toptitle=1mm, arc=.35mm, left=2mm, colframe=quarto-callout-note-color-frame, rightrule=.15mm, breakable, opacityback=0, coltitle=black, colback=white, colbacktitle=quarto-callout-note-color!10!white, opacitybacktitle=0.6, toprule=.15mm, titlerule=0mm, title=\textcolor{quarto-callout-note-color}{\faInfo}\hspace{0.5em}{Task}, bottomrule=.15mm, leftrule=.75mm]

Have a look at the clustered bar chart, and explain the pattern of the
association between \texttt{hours\ spend\ online} and \texttt{gender}.
Is the association significant? (inspect the p-value of the Pearson
Chi-Square)

\end{tcolorbox}

\begin{tcolorbox}[enhanced jigsaw, bottomtitle=1mm, toptitle=1mm, arc=.35mm, left=2mm, colframe=quarto-callout-note-color-frame, rightrule=.15mm, breakable, opacityback=0, coltitle=black, colback=white, colbacktitle=quarto-callout-note-color!10!white, opacitybacktitle=0.6, toprule=.15mm, titlerule=0mm, title=\textcolor{quarto-callout-note-color}{\faInfo}\hspace{0.5em}{Task}, bottomrule=.15mm, leftrule=.75mm]

Based on the Phi value, assess the strength the association between
\texttt{hours\ spend\ online} and \texttt{gender}?

\end{tcolorbox}

\begin{figure}

{\centering \includegraphics[width=0.7\textwidth,height=\textheight]{images/lab3_pic3_es.png}

}

\caption{Effect size}

\end{figure}

\begin{tcolorbox}[enhanced jigsaw, bottomtitle=1mm, toptitle=1mm, arc=.35mm, left=2mm, colframe=quarto-callout-important-color-frame, rightrule=.15mm, breakable, opacityback=0, coltitle=black, colback=white, colbacktitle=quarto-callout-important-color!10!white, opacitybacktitle=0.6, toprule=.15mm, titlerule=0mm, title=\textcolor{quarto-callout-important-color}{\faExclamation}\hspace{0.5em}{What a Phi value tells us}, bottomrule=.15mm, leftrule=.75mm]

Phi tells us about the strength of an association. Its value ranges from
0 to 1. Cohen (1988) provides the following guideline:

\begin{itemize}
\item
  0.1 is considered small
\item
  0.3 medium
\item
  0.5 large
\end{itemize}

\end{tcolorbox}

\hypertarget{post-hoc-test}{%
\subsection{Post-Hoc Test}\label{post-hoc-test}}

If the association were significant, you would want to know further
which group comparisons are actually significantly different. This is
called post-hoc tests or multiple comparison tests. In essence, you want
to test all pairwise comparisons to detect where the significant occurs.
For example, is the proportion of male vs female significantly different
within ``less than 1-hour'', within ``1-5 hours'' group, etc?

\textbf{How?}

Use the menu options \textbf{Analyze}\(\rightarrow\)\textbf{Descriptive
statistics}\(\rightarrow\)\textbf{Crosstabs}. Insert \texttt{gender}
into ``Columns'' and \texttt{q1\_online} into ``rows''. Click
\texttt{Cells} tab on the left, then \texttt{Tick}Row
percentages\texttt{.\ Under\ z-test,\ tick}Compare columns
proportions\texttt{and}Adjust p-values (Bonferroni method)`.

What does SPSS do with ``Adjust p-values''? The answer is that SPSS will
conduct z-tests for each comparison. Because of conducting multiple
comparisons tests, the significance level at each test should not be 5\%
anymore. It should be divided by the number of comparisons.~For example,
if you have 6 pairwise comparisons, the sig. level at each test would be
0.05/6=0.008. This is to make sure that all comparisons tests were
maintained at alpha=5\%. If you do not adjust it and use alpha=5\% in
each of the pairwise tests, the probability of getting at least one
significant result due to chance is high.

\hypertarget{post-hoc-test-output}{%
\subsection{Post-Hoc Test Output}\label{post-hoc-test-output}}

The output of the post-hoc test is given below

\begin{figure}

{\centering \includegraphics{images/lab3_post_out_revised.png}

}

\caption{Output of the post-hoc test}

\end{figure}

\begin{tcolorbox}[enhanced jigsaw, bottomtitle=1mm, toptitle=1mm, arc=.35mm, left=2mm, colframe=quarto-callout-note-color-frame, rightrule=.15mm, breakable, opacityback=0, coltitle=black, colback=white, colbacktitle=quarto-callout-note-color!10!white, opacitybacktitle=0.6, toprule=.15mm, titlerule=0mm, title=\textcolor{quarto-callout-note-color}{\faInfo}\hspace{0.5em}{Task}, bottomrule=.15mm, leftrule=.75mm]

Looking at the output of the post-hoc test, what do you conclude?

\end{tcolorbox}

\newpage{}

\hypertarget{two-independent-samples-t-test}{%
\section{Two Independent Samples T
Test}\label{two-independent-samples-t-test}}

This test allows you to explore differences in mean values across two
groups e.g., between low vs.~high-income groups; male vs.~female
consumers; those who received a reward vs.~nothing; those who get a flu
jab vs.~placebo, country A vs.~country B, etc.

Now, we want to assess if there are differences in the Opinion of
Leadership (average scores of \texttt{q10\_op1}, \texttt{q10\_op2},
\texttt{q10\_op3}) across gender (\texttt{q14\_gender}).

As in the previous lab, you need to conduct reliability analysis to find
out whether the three items of Opinion Leadership get well together, if
yes, then you need to create a composite score by taking the average of
the three items (I would suggest that you name the new variable as
\texttt{Opinion\_avg}).

In this lab, you will test following hypothesis:

H0: Opinion leadership of male consumers = Opinion leadership of female
consumers.

H1: Opinion leadership of male consumers \textgreater{} Opinion
leadership of female consumers.

\begin{tcolorbox}[enhanced jigsaw, bottomtitle=1mm, toptitle=1mm, arc=.35mm, left=2mm, colframe=quarto-callout-note-color-frame, rightrule=.15mm, breakable, opacityback=0, coltitle=black, colback=white, colbacktitle=quarto-callout-note-color!10!white, opacitybacktitle=0.6, toprule=.15mm, titlerule=0mm, title=\textcolor{quarto-callout-note-color}{\faInfo}\hspace{0.5em}{Task}, bottomrule=.15mm, leftrule=.75mm]

The above alternative hypothesis is directional. Formulate a
non-directional alternative hypothesis instead.

\end{tcolorbox}

\begin{tcolorbox}[enhanced jigsaw, bottomtitle=1mm, toptitle=1mm, arc=.35mm, left=2mm, colframe=quarto-callout-warning-color-frame, rightrule=.15mm, breakable, opacityback=0, coltitle=black, colback=white, colbacktitle=quarto-callout-warning-color!10!white, opacitybacktitle=0.6, toprule=.15mm, titlerule=0mm, title=\textcolor{quarto-callout-warning-color}{\faExclamationTriangle}\hspace{0.5em}{Warning}, bottomrule=.15mm, leftrule=.75mm]

Note that when using a two independent samples test, the grouping
variable (e.g., gender) is a qualitative variable -- nominal or ordinal,
and the test variable (e.g., amount money spent, scores on opinion
leadership) is always a quantitative variable -- interval or ratio.

\end{tcolorbox}

To conduct a T test, use the menu options
\textbf{Analyze}\(\rightarrow\)\textbf{Compare
means}\(\rightarrow\)\textbf{Independent- Samples T test}. The test
variables will be \texttt{Opinion\_avg}. The grouping variable is
\texttt{gender}. You need to ``define groups'' this means telling SPSS
that gender is measured using the numbers 1 and 2 (you will see from the
data view that \texttt{loyalty\ card\ status} is coded as 1 and 2). By
default, `Estimates effect size' was ticked'.

\begin{figure}

{\centering \includegraphics[width=5.04167in,height=\textheight]{images/lab3_2ttest_win.png}

}

\end{figure}

\hypertarget{outputs}{%
\subsection{Outputs}\label{outputs}}

In the output window you will have three tables, the first giving the
``Group Statistics'', the second giving the results of the ``Independent
Samples Test'' (The important column is the sig. columns which give the
p-values.), and the third gives the estimate of the effect size.

First, we need to look at the descriptive statistics to get a clear
picture of the mean score of female vs.~male on \texttt{Opinion\_avg},
and assess the difference between the means.

\begin{figure}

{\centering \includegraphics[width=5.04167in,height=\textheight]{images/lab3_2ttest_des.png}

}

\end{figure}

\begin{tcolorbox}[enhanced jigsaw, bottomtitle=1mm, toptitle=1mm, arc=.35mm, left=2mm, colframe=quarto-callout-note-color-frame, rightrule=.15mm, breakable, opacityback=0, coltitle=black, colback=white, colbacktitle=quarto-callout-note-color!10!white, opacitybacktitle=0.6, toprule=.15mm, titlerule=0mm, title=\textcolor{quarto-callout-note-color}{\faInfo}\hspace{0.5em}{Task}, bottomrule=.15mm, leftrule=.75mm]

Look at the descriptive statistic above, say something about the mean
scores and their standard deviations.

\end{tcolorbox}

Now, let us discuss the second table.

\begin{figure}

{\centering \includegraphics[width=8.16667in,height=\textheight]{images/lab3_2ttest_levene.png}

}

\end{figure}

The second column in the table gives the significance value for Levene's
test (\textbf{Block 1}). This tells you if you can assume that the
variance for the groups is equal, which cannot be assumed for this
example and therefore when interpreting the result of the T-tests you
should look at the p-value for ``equal variances not assumed'' (bottom
row of \textbf{Block 2}). The p-value associated with the one-sided test
was circled because our alternative hypothesis was directional (the
males' scores is higher than the females' scores).

\begin{tcolorbox}[enhanced jigsaw, bottomtitle=1mm, toptitle=1mm, arc=.35mm, left=2mm, colframe=quarto-callout-note-color-frame, rightrule=.15mm, breakable, opacityback=0, coltitle=black, colback=white, colbacktitle=quarto-callout-note-color!10!white, opacitybacktitle=0.6, toprule=.15mm, titlerule=0mm, title=\textcolor{quarto-callout-note-color}{\faInfo}\hspace{0.5em}{Task}, bottomrule=.15mm, leftrule=.75mm]

The one-sided p-value in the second block is significant (circled in
green), what do you conclude?

\end{tcolorbox}

\hypertarget{is-the-difference-practically-meaningful}{%
\subsection{Is the Difference Practically
Meaningful?}\label{is-the-difference-practically-meaningful}}

If the mean difference is significant. We should ask if the different is
meaningful or not. To answer this question, we need to assess the effect
size table.

\emph{What is effect size?} Imagine that if you took a paracetamol when
you had a migraine and the pill reduces the pain slightly but it is not
really helping you much -- so the effect of the pill is small.

\begin{tcolorbox}[enhanced jigsaw, bottomtitle=1mm, toptitle=1mm, arc=.35mm, left=2mm, colframe=quarto-callout-note-color-frame, rightrule=.15mm, breakable, opacityback=0, coltitle=black, colback=white, colbacktitle=quarto-callout-note-color!10!white, opacitybacktitle=0.6, toprule=.15mm, titlerule=0mm, title=\textcolor{quarto-callout-note-color}{\faInfo}\hspace{0.5em}{Guidelines on the effect size of mean difference (Cohen 1988)}, bottomrule=.15mm, leftrule=.75mm]

d = 0.2 is considered small

0.5 medium

0.8 large

For example, if the difference between two groups' means \textless{} 0.2
standard deviations, despite the difference being significant, it is not
practically meaningful.

\end{tcolorbox}

Let us look at the effect size output below:

\begin{figure}

{\centering \includegraphics[width=6.08333in,height=\textheight]{images/lab3_2ttest_es.png}

}

\end{figure}

\begin{tcolorbox}[enhanced jigsaw, bottomtitle=1mm, toptitle=1mm, arc=.35mm, left=2mm, colframe=quarto-callout-note-color-frame, rightrule=.15mm, breakable, opacityback=0, coltitle=black, colback=white, colbacktitle=quarto-callout-note-color!10!white, opacitybacktitle=0.6, toprule=.15mm, titlerule=0mm, title=\textcolor{quarto-callout-note-color}{\faInfo}\hspace{0.5em}{Task}, bottomrule=.15mm, leftrule=.75mm]

Look at the Cohen's d in the above table, what does it tell you?

\end{tcolorbox}

\hypertarget{interpretation-of-the-findings}{%
\subsection{Interpretation of the
Findings}\label{interpretation-of-the-findings}}

Looking at the descriptive statistics and statistical results, we can
conclude that there are statistically significant differences between
male and female consumers on their ratings on opinion leadership.
Specifically, male consumers have a higher rating of opinion leadership
compared to that of female consumers (\(M_{Male} = 4.66\)
vs.~\(M_{Female} = 3.77\), Cohen's d = 0.45). Furthermore in terms of
the variability in the opinion leadership across the two groups (male
and female consumers), the Levene's test show that there are differences
in the spread, that is, the variability in the opinion of leadership is
not the same for both groups (\(SD_{Male} = 1.87\)
vs.~\(SD_{Female} = 2.03\)).

\begin{tcolorbox}[enhanced jigsaw, bottomtitle=1mm, toptitle=1mm, arc=.35mm, left=2mm, colframe=quarto-callout-note-color-frame, rightrule=.15mm, breakable, opacityback=0, coltitle=black, colback=white, colbacktitle=quarto-callout-note-color!10!white, opacitybacktitle=0.6, toprule=.15mm, titlerule=0mm, title=\textcolor{quarto-callout-note-color}{\faInfo}\hspace{0.5em}{Task}, bottomrule=.15mm, leftrule=.75mm]

Why not investigating if there are differences in the satisfaction level
between those who are willing to participate in Dell loyalty program and
those who are not.

\end{tcolorbox}

\hypertarget{video-1}{%
\section{Video}\label{video-1}}

\href{https://dtu-panopto.lancs.ac.uk/Panopto/Pages/Viewer.aspx?id=e0c9c7c6-9576-45fd-9094-acab0114a359}{One-sample
T-test}

\href{https://dtu-panopto.lancs.ac.uk/Panopto/Pages/Viewer.aspx?id=95a76df2-502e-4267-8a46-acab010a11ac}{Two
independent samples T-test}

\href{https://dtu-panopto.lancs.ac.uk/Panopto/Pages/Viewer.aspx?id=888c210d-d357-4207-9736-acab0129b6c6}{Crosstabs}

\hypertarget{readings-1}{%
\section{Readings}\label{readings-1}}

Feick, L. F., \& Price, L. L. (1987). The market maven: A diffuser of
marketplace information. \emph{Journal of Marketing}, 51(1), 83-97.

Goldsmith, R. E., Flynn, L. R., \& Goldsmith, E. B. (2003). Innovative
consumers and market mavens. \emph{Journal of Marketing Theory and
Practice}, 11(4), 54-65.

\bookmarksetup{startatroot}

\hypertarget{sec-lab4}{%
\chapter{ANOVA and Experimentation (Week 16)}\label{sec-lab4}}

Data: MBGshort.sav

Data is available on `Workshop Materials' folder on Moodle.

\hypertarget{learning-objectives-3}{%
\section{Learning objectives}\label{learning-objectives-3}}

The aim of this lab is to help you to use SPSS to analyze data from an
experimentation. Specifically, we want to examine mean differences
across three experimental groups (one-way ANOVA design) and analyze data
of a more complex experiment (i.e., two-way ANOVA in the form of a 3x2
between-subject design).

Learning objectives: At the end of this lab, we hope that you will be
able to

\begin{itemize}
\item
  Analyze data from a one-way ANOVA (Analysis of Variance) experiment.
\item
  Produce and interpret basic SPSS outputs from a two-way ANOVA
  experiment.
\end{itemize}

In this lab, we are going to look at how one-way ANOVA can be used to
extend the two independent samples T test to look for means differences
across three or more groups and secondly to show how data resulted from
manipulation of more than one factor can be analyzed by extending
one-way ANOVA method. For instance, to look at how types of return
condition facilitated by money-back guarantees (MBG for short) (No MBG,
15 days, 30 days) and types of product (search vs.~experience product)
influence perceived product quality. We focus our attention to
between-subject ANOVA designs where each respondent is randomly assigned
to only one of the experimental conditions.

\hypertarget{mean-differences-across-three-or-more-groups}{%
\section{Mean Differences Across Three or More
Groups}\label{mean-differences-across-three-or-more-groups}}

One-way ANOVA can also be used to explore differences in a variable
across three or more groups. This is more useful than the two
independent sample T test simply because it can be used for more groups
and can tell us where the location of the differences. For instance you
might find significant mean differences on perceived product quality
across three groups (e.g., No MBG, 15 days, and 30 days) and to be more
specific, the 15 days and 30 days group do not differ. But perceived
product quality of participants in the No MBG group is different than
those in the 15 days and 30 days group.

In this part, we are going to use the data collected by a former
Advanced Marketing Management student at Lancaster University for his
dissertation about the effect of MBG on perceived durability of a
product. He created an experiment by devising three different scenarios
where each scenarios contains information about each of MBG conditions
(No MBG, 15 days, 30 days). Respondents were randomly assigned to read
one of three different questionnaires. In each questionnaire, he put an
image of a product (he chose a laptop) and information about the MBG
condition. Other information across different questionnaires was kept
similar (e.g., product specification, laptop price, etc).

\begin{figure}

{\centering \includegraphics{images/lab4_laptop.png}

}

\caption{An example of stimulus used in the 30 days MBG condition. This
image was created by William, a former AMM student for his MSc
dissertation}

\end{figure}

Let us explore if the difference exist in the perceived product
durability across the three groups MBG. We will use quality5 (i.e., `the
product of this offer would be likely to be durable') as the dependent
variable.

To conduct an ANOVA, use the menu options
\textbf{Analyze}\(\rightarrow\)Compare means\(\rightarrow\)One-way
ANOVA. Click \texttt{Options} then tick \texttt{Descriptive}. You can
also tick \textbf{Options}\(\rightarrow\)means plot to display a graph
that shows the means of the three groups Click \texttt{Continue} then
\texttt{OK}. The dependent variable is \texttt{quality5} (i.e., `the
product of this offer would be likely to be durable'). The factor is
\texttt{tMBG}.

In the output window the SPSS produces the Descriptive table and ANOVA
table.

\includegraphics{images/lab4_oneway_des.png}

\includegraphics{images/lab4_oneway_sig.png}

The ANOVA table reveals that perceived product durability differ across
the experimental groups because the sig-value is less than 5\%. The
descriptive table gives an indication of why the differences exist. No
MBG group has a lower mean compared the 15-days and 30-days group. The
15-days and 30-days groups are similar. However, you need to do
additional test to get more details about the differences. Let's try
again putting some important options.

\hypertarget{homogeneity-of-variance-and-post-hoc-tests}{%
\section{Homogeneity of Variance and Post-Hoc
Tests}\label{homogeneity-of-variance-and-post-hoc-tests}}

\hypertarget{homogeneity-of-variance-test}{%
\subsection{Homogeneity of Variance
Test}\label{homogeneity-of-variance-test}}

Repeat the analysis above and click the \texttt{Options} tab. Check on
\texttt{Options} and tick \texttt{Homogeneity\ of\ Variance\ test}. To
determine which of the three groups differ, you can do what is called a
Post Hoct test - click \texttt{Post\ Hoc}. There are many tests
available. Among many options, check \texttt{Scheffe} for situation
where Equal Variances Assumed; check \texttt{Games-Howell} under Equal
Variances Not Assumed.

Homogeneity variances test determines whether you would use Scheffe or
Games-Howell test. You use this rule:

\begin{tcolorbox}[enhanced jigsaw, bottomtitle=1mm, toptitle=1mm, arc=.35mm, left=2mm, colframe=quarto-callout-tip-color-frame, rightrule=.15mm, breakable, opacityback=0, coltitle=black, colback=white, colbacktitle=quarto-callout-tip-color!10!white, opacitybacktitle=0.6, toprule=.15mm, titlerule=0mm, title=\textcolor{quarto-callout-tip-color}{\faLightbulb}\hspace{0.5em}{Tip}, bottomrule=.15mm, leftrule=.75mm]

If sig-value \textless{} 0.05, use Games and Howell, otherwise use
Scheffe

\end{tcolorbox}

\begin{figure}

{\centering \includegraphics{images/lab4_oneway_levene.png}

}

\end{figure}

The sig-value of the homogeneity of variance test is 0.906, therefore
you focus on the outputs of the Scheffe test in the next table. Let us
interpret the results of the Scheffe test.

\hypertarget{post-hoc-test-1}{%
\subsection{Post Hoc Test}\label{post-hoc-test-1}}

In the SPSS outputs, you have the following table

\begin{figure}

{\centering \includegraphics{images/lab4_oneway_posthoc.png}

}

\caption{Post Hoc Tests}

\end{figure}

You can see from the above table that No MBG is not statistically
different from 15 days (p = 0.08) and 15 days is not statistically
different from the 30 days (p = 0.766). But no MBG has a statistical
difference with the 30 days (p = 0.013) (the p-value between No MBG and
15 days are close to significant, if sample size is large, it would be
likely to be significant). The next output from the Scheffe test below
clarifies the difference.

\begin{figure}

{\centering \includegraphics{images/lab4_oneway_sch_hom.png}

}

\caption{Homogenous Subsets}

\end{figure}

\hypertarget{two-way-anova-experiment}{%
\section{Two-way ANOVA Experiment}\label{two-way-anova-experiment}}

The student thought that the results might not be the same if other
product is used. He collected new data but use cloth (i.e, T-shirt) as
the focal product. In the data, he created a new variable \texttt{tGood}
and coded it as 0 for laptop and 1 for cloth.

Laptop is an example of search goods where consumers can fully search
for information about the attributes of the product prior to purchase.
In contrast, cloth is an example of experience goods where consumers can
only acquire limited information without their direct experiences -- to
know whether or not a cloth fits you well then you have to wear it!

\begin{tcolorbox}[enhanced jigsaw, bottomtitle=1mm, toptitle=1mm, arc=.35mm, left=2mm, colframe=quarto-callout-note-color-frame, rightrule=.15mm, breakable, opacityback=0, coltitle=black, colback=white, colbacktitle=quarto-callout-note-color!10!white, opacitybacktitle=0.6, toprule=.15mm, titlerule=0mm, title=\textcolor{quarto-callout-note-color}{\faInfo}\hspace{0.5em}{Note}, bottomrule=.15mm, leftrule=.75mm]

The experiment above can be described in shorthand notation as a 3 (type
of MBG: No MBG, 15 days, 30 days) X 2 (type of product: search good
vs.~experience good) between-subject ANOVA design or 3 X 2
Between-Subjects ANOVA, for short. X is read as ``by''. Between-subject
is another term for randomization, which means that participants were
randomly assigned to experimental groups. Because there are two
variables being manipulated (type of MBG, type product), in general the
design is referred to as a \textbf{two-way ANOVA} design.

\end{tcolorbox}

\begin{tcolorbox}[enhanced jigsaw, bottomtitle=1mm, toptitle=1mm, arc=.35mm, left=2mm, colframe=quarto-callout-note-color-frame, rightrule=.15mm, breakable, opacityback=0, coltitle=black, colback=white, colbacktitle=quarto-callout-note-color!10!white, opacitybacktitle=0.6, toprule=.15mm, titlerule=0mm, title=\textcolor{quarto-callout-note-color}{\faInfo}\hspace{0.5em}{Note}, bottomrule=.15mm, leftrule=.75mm]

Because there are two variables being manipulated (type of MBG, type of
product), the design is classified as a \textbf{two-way ANOVA} design.

\end{tcolorbox}

To analyze data from this experiment, use the menu options
\textbf{Analyze}\(\rightarrow\)\textbf{General Linear Model}
\(\rightarrow\)\textbf{Univariate}. Assign \texttt{quality5} as the
dependent variable and \texttt{tMBG} and \texttt{tGood} as the
independent variables. It would also be useful to plot the relationship
between the variables. So click on the plot tab and specify
\texttt{tMBG} as the horizontal axis and \texttt{tGood} as the separate
lines. Also in order to see the means for each group you need to click
on the \texttt{Options} tab and check \texttt{Descriptive\ statistics}.
The following tables provide descriptive statistics for each
experimental group.

\begin{figure}

{\centering \includegraphics{images/lab4_2way_des.png}

}

\caption{Cell frequency: This gives the number of respondents within
each experimental condition}

\end{figure}

\begin{figure}

{\centering \includegraphics{images/lab4_2way_des2.png}

}

\caption{This table provides the mean for each group}

\end{figure}

The most important table is the following table.
\includegraphics[width=0.8\textwidth,height=\textheight]{images/lab4_2way_sig.png}

The previous table tests us if there are differences due to type of MBG
condition and type of product under evaluations. It also tells us if
there is an interaction between the two factors. We should interpret the
sig-value as before, with a sig-value less than 0.05 indicating that
there are statistically significant mean differences across levels of
factors.

The table shows us that \texttt{tMBG} is significant
(sig-value=0.006\textless0.05). This means that respondents have
different perception regarding the durability of the products across the
three levels of MBG regardless of product types. \texttt{tGood} is not
significant (sig-value=0.539), which means that respondents have similar
perceptions about the product durability regardless of types of MBG
offered. You can say, these situations reflect the presence of the main
effect of \texttt{tMBG} but not \texttt{tGood}.

The most interesting findings from the above table is the sig-value
related to the expression tMBG*tGood, which reflects the interaction
effect between \texttt{tMBG} and \texttt{tGood}. The interaction effect
means that the effect of \texttt{tMBG} is different across levels of
\texttt{tGood}. The meaning will become clearer (hopefully!) when you
see the plots of the means below.

\begin{figure}

{\centering \includegraphics{images/lab4_2way_barchrt.png}

}

\caption{Bar chart showing means}

\end{figure}

As you can see from the figure above, in the No MBG and 15 days, the
perceived durability of search good is higher than that of the
experience good. But when the MBG is 30days, the perceived durability of
search good is lower than that of the experience good.

Line graph is also often used to plot the interaction.

\begin{figure}

{\centering \includegraphics{images/lab4_2way_linegrp.png}

}

\caption{Line graph showing an interaction effect exists}

\end{figure}

\begin{tcolorbox}[enhanced jigsaw, bottomtitle=1mm, toptitle=1mm, arc=.35mm, left=2mm, colframe=quarto-callout-note-color-frame, rightrule=.15mm, breakable, opacityback=0, coltitle=black, colback=white, colbacktitle=quarto-callout-note-color!10!white, opacitybacktitle=0.6, toprule=.15mm, titlerule=0mm, title=\textcolor{quarto-callout-note-color}{\faInfo}\hspace{0.5em}{Task}, bottomrule=.15mm, leftrule=.75mm]

What do you think about the potential managerial implications of the
above findings?

\end{tcolorbox}

\hypertarget{learning-experimentation-from-published-research}{%
\section{Learning Experimentation from Published
Research}\label{learning-experimentation-from-published-research}}

You can learn experimentation from published research. You can even
replicate results of published studies as more authors nowadays made
their data publicly available as a way to increase research
transparency. Open Science Forum \url{https://osf.io} facilitates such
initiatives. You can click this \href{https://osf.io/j6bns}{link} or
this
\href{https://osf.io/tjgma/?view_only=d9cb35f5c1984edaac208aa90a072ec5}{link}
as examples of such documentation.

\hypertarget{video-2}{%
\section{Video}\label{video-2}}

Lecture Week 15 on Experimentation

\href{https://dtu-panopto.lancs.ac.uk/Panopto/Pages/Viewer.aspx?id=4b6ee74f-4aae-4b7f-aa3b-af9c006b039e}{2023}
\href{https://lancaster.cloud.panopto.eu/Panopto/Pages/Viewer.aspx?id=1a268e6d-703f-4fee-ba1d-b11000598cc5}{2024}

\href{https://dtu-panopto.lancs.ac.uk/Panopto/Pages/Viewer.aspx?id=4f7d12dd-f094-46c6-8905-acab010a1756}{One-way
ANOVA}

\href{https://dtu-panopto.lancs.ac.uk/Panopto/Pages/Viewer.aspx?id=d3d18d3d-b867-4772-8123-ac9001049583}{Analyzing
data from two-way ANOVA}

\bookmarksetup{startatroot}

\hypertarget{sec-lab5}{%
\chapter{Regression (Week 17)}\label{sec-lab5}}

Data: RobotGLP.sav

Data is available on `Workshop Materials' folder on Moodle.

\hypertarget{learning-objectives-4}{%
\section{Learning objectives}\label{learning-objectives-4}}

The aim of this lab is to help you to use SPSS to conduct regression
analysis, which is useful in explaining the relationship between a set
of \emph{independent} variables and a \emph{dependent variable}.

At the end of this lab, we hope that you will be able to

\begin{itemize}
\item
  Understand the meaning of independent and dependent variables
\item
  Select appropriate independent variables to explain a dependent
  variable
\item
  Produce and interpret basic SPSS outputs for multiple regression
\item
  Understand the meaning of multicollinearity, how to detect and remedy
  it
\end{itemize}

\begin{tcolorbox}[enhanced jigsaw, bottomtitle=1mm, toptitle=1mm, arc=.35mm, left=2mm, colframe=quarto-callout-important-color-frame, rightrule=.15mm, breakable, opacityback=0, coltitle=black, colback=white, colbacktitle=quarto-callout-important-color!10!white, opacitybacktitle=0.6, toprule=.15mm, titlerule=0mm, title=\textcolor{quarto-callout-important-color}{\faExclamation}\hspace{0.5em}{Important}, bottomrule=.15mm, leftrule=.75mm]

FYI, many students of the previous cohorts used regression when they
wrote their MSc dissertations.

\end{tcolorbox}

\hypertarget{why-regression}{%
\section{Why Regression}\label{why-regression}}

In marketing research, we often need to determine the impact of a set of
marketing variables on one variable -- a factor we want to understand or
predict. Furthermore, among those variables we may want to find out
which variables matter most, and which variables are not so important
that we can ignore. Regression analysis can help us findings the answers
to these questions.

In this workshop, we want to investigate factors that influence
consumers' decision to join a green loyalty program -- This LP
encourages behaviours from hotel guests that are good for environment
(e.g., reuse towels).

\begin{figure}

\includegraphics[width=0.5\textwidth,height=\textheight]{images/lab5_robot.jpg} \hfill{}

\caption{Hotel receptionist (Image courtesy of Jingxi)}

\end{figure}

In this lab, we consider the following case. A hotel manager who
recently launch a green loyalty program (GLP) wants to know about
factors that affect consumers' intention to join the hotel GLP. The
hotel manager decides to develop a survey and asks the hotel's guests to
fill in an offline survey. The survey form was handed in to hotel guests
by a frontline service robot (see the image above). The survey contains
items to measure the following constructs:

\begin{itemize}
\item
  Intention to join the green loyalty program
\item
  Anticipated guilt if not joining the green loyalty program
\item
  Perceived attractiveness of hotel receptionist
\item
  Age
\item
  Gender
\end{itemize}

\texttt{Anticipated\ guilt}, \texttt{Perceived\ Attractiveness},
\texttt{Age}, \texttt{Gender} are called the independent variables and
\texttt{Intention\ to\ join\ the\ green\ loyalty\ program} is the
dependent variable. The notion of the dependent variable comes from our
prediction that its value depends on the values of the independent
variables. The relationship between
\texttt{Intention\ to\ join\ the\ green\ LP} the and the four
independent variables can be written as:

Intention to join the green LP \textasciitilde{} Anticipated Guilt +
Perceived Attractiveness + Age + Gender + error.

You can read the above expression as a consumer' intention to join the
green loyalty program is influenced by the consumer's anticipated of
guilt, perceived attractiveness, age, and gender, and some unknown
factors represented by an error term. The relationship can be
represented by a mathematical expression as below. The names of the
variables are shortened.

\begin{equation}
Intent = \beta_0 +  \beta_1 * Guilt + \beta_2 * Attract + \beta_3 * Age + \beta_4 * Gender.
\end{equation}

where \(\beta_0\), \(\beta_1\), \(\beta_2\), \(\beta_3\), and
\(\beta_4\) are parameters that capture the impact of each of the
independent variables on \texttt{intention}.

We use regression procedure in SPSS to find out the estimates for all
parameters using sample data (\(\beta_0\) is just a constant so it is
not of our interest).

\begin{tcolorbox}[enhanced jigsaw, bottomtitle=1mm, toptitle=1mm, arc=.35mm, left=2mm, colframe=quarto-callout-important-color-frame, rightrule=.15mm, breakable, opacityback=0, coltitle=black, colback=white, colbacktitle=quarto-callout-important-color!10!white, opacitybacktitle=0.6, toprule=.15mm, titlerule=0mm, title=\textcolor{quarto-callout-important-color}{\faExclamation}\hspace{0.5em}{Important}, bottomrule=.15mm, leftrule=.75mm]

Variable \texttt{Intent}, \texttt{Guilt}, and \texttt{Attract} were
average scores of a multi-item scale where each item in the scale were
measured using a Likert scale ranging from 1 = strongly disagree to
7=strongly agree. \texttt{Age} is a continuous variable, and
\texttt{Gender} is a 0,1 variable.

\end{tcolorbox}

If you want to examine whether \texttt{Guilt} is a significant factor,
then you want to test \(H0\): \(\beta_1=0\) against \(H1\):
\(\beta_1 \ne 0\). SPSS will report the p-value associated with \(H0\).
If p-value is less than 0.05, you reject \(H0\) otherwise retain it. If
you reject \(H0\), you can conclude that \texttt{Guilt} is a significant
factor that influences \texttt{intent}.

If you hypothesize that \texttt{Guilt} is a significant factor and has a
positive impact on satisfaction, then you want to test \(H0\):
\(\beta_1=0\) against \(H1\): \(\beta_1 > 0\). SPSS will report the
p-value associated with \(H0\). You have to divide the p-value by 2. If
the p-value/2 is less than 0.05, you reject \(H0\) otherwise retain it.
If you reject \(H0\), you can conclude that \texttt{Guilt} has a
significant positive influence on \texttt{Intent}.

\hypertarget{conducting-multiple-regression-with-spss}{%
\section{Conducting Multiple Regression with
SPSS}\label{conducting-multiple-regression-with-spss}}

\textbf{Open RobotGLP.sav} -- Thanks to
\href{https://www.lancaster.ac.uk/lums/people/jingxi-huang}{Jingxi} for
allowing us to use a subset of her data to test the above model.

To conduct a regression analysis in SPSS, click the following:
\textbf{Analyze}\(\rightarrow\)\textbf{Regression}\(\rightarrow\)\textbf{Linear.}

Enter \texttt{Intent} into the dependent variable box.

Enter \texttt{Guilt},\texttt{Attract}, \texttt{Age,}Gender` into the
independent variable(s) box.

\begin{figure}

{\centering \includegraphics[width=0.7\textwidth,height=\textheight]{images/lab5_ols.png}

}

\end{figure}

Click \texttt{OK}

SPSS produces four tables. The first one does not offer much
information. Therefore, we focus our attention to the next three tables.

\begin{enumerate}
\def\labelenumi{\arabic{enumi}.}
\item
  ANOVA table: Is the model meaningful?

  \includegraphics[width=0.7\textwidth,height=\textheight]{images/lab5_anova.png}
\end{enumerate}

This table tells whether or not the model is meaningful. If the sig
value is less than 0.05, then the model is meaningful. If p-value is
greater than 0.05, then model should be dismissed and don't interpret
other outputs. In this example, the sig. value is less than 0.05;
therefore we have a meaningful model. We can proceed with the next
output.

\begin{enumerate}
\def\labelenumi{\arabic{enumi}.}
\setcounter{enumi}{1}
\tightlist
\item
  Model Summary table: How good is the model?
\end{enumerate}

Having known the model is meaningful, how do we know whether it is good
enough? The Model Summary table provides the answer to this question.

\includegraphics[width=0.7\textwidth,height=\textheight]{images/lab5_rsq.png}

The model summary table tells you how well the independent variables
explain variation in the dependent variable. The adjusted \(R^2\) is
0.311 indicating that about 31\% of the variation in the intention score
is explained by the four independent variables (Guilt, Attract, Age, and
Gender). Theoretically, the maximum possible value for adjusted R-square
is 100\% indicating a perfect model!

Adjusted \(R^2\) is \(R^2\) that is adjusted for the number of
independent variables in the model. The more independent variable you
have in the model, the larger the \(R^2\) will be. The adjusted \(R^2\)
prevents the inflation. You can use the following convention to qualify
the impact of the set of the independent variables on the dependent
variable\footnote{Ellis, Paul D. 2011. The Essential Guide to Effect
  Sizes, p.41}:

\(R^2\) ≥ 0.02 small effect

≥ 0.13 medium

≥ 0.26 large

\begin{tcolorbox}[enhanced jigsaw, bottomtitle=1mm, toptitle=1mm, arc=.35mm, left=2mm, colframe=quarto-callout-note-color-frame, rightrule=.15mm, breakable, opacityback=0, coltitle=black, colback=white, colbacktitle=quarto-callout-note-color!10!white, opacitybacktitle=0.6, toprule=.15mm, titlerule=0mm, title=\textcolor{quarto-callout-note-color}{\faInfo}\hspace{0.5em}{Task}, bottomrule=.15mm, leftrule=.75mm]

Knowing the adjusted \(R^2\), indicate the strength of the impact of
Guilt, Attract, Age, and Gender.

\end{tcolorbox}

\begin{tcolorbox}[enhanced jigsaw, bottomtitle=1mm, toptitle=1mm, arc=.35mm, left=2mm, colframe=quarto-callout-warning-color-frame, rightrule=.15mm, breakable, opacityback=0, coltitle=black, colback=white, colbacktitle=quarto-callout-warning-color!10!white, opacitybacktitle=0.6, toprule=.15mm, titlerule=0mm, title=\textcolor{quarto-callout-warning-color}{\faExclamationTriangle}\hspace{0.5em}{Warning}, bottomrule=.15mm, leftrule=.75mm]

Maximizing \(R^2\) value should not be your main goal in regression. Do
not be tempted to select variables with the aim of increasing \(R^2\)
value. \(R^2\) value in social sciences are typically in the range of
0.1 to 0.5. Selecting variables should be motivated by theories.

\end{tcolorbox}

\begin{enumerate}
\def\labelenumi{\arabic{enumi}.}
\setcounter{enumi}{2}
\tightlist
\item
  Coefficients
\end{enumerate}

\includegraphics[width=1.1\textwidth,height=\textheight]{images/lab5_coefs.png}

This table tells us which of the independent variables significantly
explain or predict the dependent variable. In this case three variables
significantly explain \texttt{Intent}. These are \texttt{Guilt},
\texttt{Attract}, and \texttt{Age}, whereas \texttt{Gender} is not
significant. Furthermore, the standardized beta values tell us which
variable has the strongest impact on \texttt{Intent}. In this case
\texttt{Attract} is the strongest predictor followed by \texttt{Guilt}
and \texttt{Age}.

\begin{tcolorbox}[enhanced jigsaw, bottomtitle=1mm, toptitle=1mm, arc=.35mm, left=2mm, colframe=quarto-callout-note-color-frame, rightrule=.15mm, breakable, opacityback=0, coltitle=black, colback=white, colbacktitle=quarto-callout-note-color!10!white, opacitybacktitle=0.6, toprule=.15mm, titlerule=0mm, title=\textcolor{quarto-callout-note-color}{\faInfo}\hspace{0.5em}{Task}, bottomrule=.15mm, leftrule=.75mm]

If you want to make a prediction, use the unstandardized coefficients
(B). If someone assigns a rating of three on \texttt{Guilt},
\texttt{Attract} can you predict her intention level? (round-off your
answer)

\end{tcolorbox}

\hypertarget{multicollinearity-problems}{%
\section{Multicollinearity Problems}\label{multicollinearity-problems}}

High correlation among independent variables is problematic in multiple
regression because it is hard for us to determine the individual
contribution of each of the independent variables in the model. For
example, if \texttt{Attract} is highly correlated with \texttt{Guilt},
then we cannot conclude that \texttt{Attract} have the strongest
influence on \texttt{Intent} because \texttt{Guilt} also gives
contribution to the magnitude of Attract-Intent relationship (it looks
stronger than it should be). In other words, the coefficients associated
with the regression estimates are biased -- not the same as their true
population values.

The situation where an independent variable is highly correlated with
another variable is referred to as multicollinearity problem. To
diagnose whether multicollinearity exists, you can check it in two ways.

\begin{enumerate}
\def\labelenumi{\arabic{enumi}.}
\tightlist
\item
  Inspect the correlation coefficient among the independent variables.
  To do correlation analysis in SPSS, follow this step:
  \textbf{Analyze}\(\rightarrow\)\textbf{Correlation}\(\rightarrow\)\textbf{Bivariate}.
  If two variables is highly correlated with the correlation coefficient
  is larger than 0.9, your model may suffer from a multicollinearity
  problem. In practice, you should be cautious, if the correlation is
  0.7.
\end{enumerate}

\begin{tcolorbox}[enhanced jigsaw, bottomtitle=1mm, toptitle=1mm, arc=.35mm, left=2mm, colframe=quarto-callout-note-color-frame, rightrule=.15mm, breakable, opacityback=0, coltitle=black, colback=white, colbacktitle=quarto-callout-note-color!10!white, opacitybacktitle=0.6, toprule=.15mm, titlerule=0mm, title=\textcolor{quarto-callout-note-color}{\faInfo}\hspace{0.5em}{Task}, bottomrule=.15mm, leftrule=.75mm]

Check the correlation coefficients among the predictors in the model.

\end{tcolorbox}

\begin{enumerate}
\def\labelenumi{\arabic{enumi}.}
\setcounter{enumi}{1}
\tightlist
\item
  Rerun your regression.
\end{enumerate}

\begin{itemize}
\item
  Click:
  \textbf{Analyze}\(\rightarrow\)\textbf{Regression}\(\rightarrow\)\textbf{Linear}.
\item
  Click \texttt{Statistics} options.
\item
  Select \texttt{Collinearity\ diagnostic}. Click \texttt{Continue} and
  Click \texttt{OK}.
\item
  Inspect the \textbf{VIF (Variance Inflation Factor} values in the
  output.
\end{itemize}

\begin{tcolorbox}[enhanced jigsaw, bottomtitle=1mm, toptitle=1mm, arc=.35mm, left=2mm, colframe=quarto-callout-tip-color-frame, rightrule=.15mm, breakable, opacityback=0, coltitle=black, colback=white, colbacktitle=quarto-callout-tip-color!10!white, opacitybacktitle=0.6, toprule=.15mm, titlerule=0mm, title=\textcolor{quarto-callout-tip-color}{\faLightbulb}\hspace{0.5em}{Tip}, bottomrule=.15mm, leftrule=.75mm]

Multicollinearity exists if VIF (variance inflation factor)
\textgreater{} 10.

\end{tcolorbox}

\includegraphics[width=1.1\textwidth,height=\textheight]{images/lab5_vif.png}

\begin{tcolorbox}[enhanced jigsaw, bottomtitle=1mm, toptitle=1mm, arc=.35mm, left=2mm, colframe=quarto-callout-note-color-frame, rightrule=.15mm, breakable, opacityback=0, coltitle=black, colback=white, colbacktitle=quarto-callout-note-color!10!white, opacitybacktitle=0.6, toprule=.15mm, titlerule=0mm, title=\textcolor{quarto-callout-note-color}{\faInfo}\hspace{0.5em}{Task}, bottomrule=.15mm, leftrule=.75mm]

See the above output, does the model suffer from a multicollinearity
problem?

\end{tcolorbox}

\hypertarget{if-multicollinearity-exists}{%
\section{If Multicollinearity
Exists}\label{if-multicollinearity-exists}}

What should you do if multicollinearity exists?

You can deal with it using one of the following techniques:

\begin{enumerate}
\def\labelenumi{\arabic{enumi}.}
\item
  Omit one or more highly correlated independent variables
\item
  Create a composite variable e.g., by taking the average score of two
  variables if the two variables causes multicollinearity
  \textbf{averaging} or using Principle Component Analysis (beyond the
  scope of our workshops)
\item
  Use the model that suffers from multicollinearity but for prediction
  purposes only.
\item
  Collect more samples
\end{enumerate}

\hypertarget{heteroskedasticity-problem}{%
\section{Heteroskedasticity Problem}\label{heteroskedasticity-problem}}

One of the assumptions in regression is that residuals or errors should
be constant across any values of independent variables. This is referred
to as the \textbf{homoskedasticity} assumption. The opposite of
homoskedasticity is \textbf{heteroskedasticity} where residuals are not
constant (i.e., heteroskedastic errors) (\emph{learn how to pronounce
these words, it took me a while to get used to them!}).

If heteroskedasticity exists, hypotheses tests about the regression
parameters are not correct anymore. See lecture slide for more details.
One of the recommended techniques to handle heteroskedasticity is to
adjust the standard errors of the regression estimates (i.e., values in
the \texttt{Std.\ error} column in the SPSS output).

\begin{tcolorbox}[enhanced jigsaw, bottomtitle=1mm, toptitle=1mm, arc=.35mm, left=2mm, colframe=quarto-callout-note-color-frame, rightrule=.15mm, breakable, opacityback=0, coltitle=black, colback=white, colbacktitle=quarto-callout-note-color!10!white, opacitybacktitle=0.6, toprule=.15mm, titlerule=0mm, title=\textcolor{quarto-callout-note-color}{\faInfo}\hspace{0.5em}{Task}, bottomrule=.15mm, leftrule=.75mm]

Rerun your regression with heteroskedasticity-adjusted standard errors
and compare the results with the original regression you had previously
conducted. Continue reading the text below.

\end{tcolorbox}

To rerun your regression, install and use the `HeteroskedasticityV3.spd'
macro developed by yours truly\footnote{Daryanto, A. (2020). Tutorial on
  heteroskedasticity using heteroskedasticityV3 SPSS macro. The
  Quantitative Methods for Psychology, 16(5), 8-20}. The macro produces
regression outputs with/without heteroscedasticity-adjusted standard
errors.

\hypertarget{sec-install}{%
\section{Installing the HeteroskedasticityV3 Macro}\label{sec-install}}

SPSS in the Lab PCs or from cloud may not permit you to install the
add-on macro because you do not have an admin right to do so. However,
it is worth trying to install the macro, it may work! Alternatively, you
need to install the macro in your personal PC where you have an admin
right to install it.

You can download the macro from these websites:

\begin{itemize}
\item
  \url{https://github.com/ahmaddaryanto/Heteroskedasticity}
\item
  Click the green button \texttt{Code}, and select
  \texttt{Download\ ZIP}.
\item
  Go to the download folder on your PC or laptop.
\item
  Locate the HeteroskedasticityV3.spd and install.
\item
  If you have an admin right, double click the file, and follow the
  instruction on your screen
\item
  If the above does not work, click
  \textbf{Extensions}\(\rightarrow\)\textbf{Utilities}\(\rightarrow\)\textbf{Install
  Custom Dialog}. Locate the file, and install.
\end{itemize}

\begin{figure}

{\centering \includegraphics{images/lab5_macro_path.png}

}

\caption{HeteroskedaticityV3 Macro in the SPSS menu after installation}

\end{figure}

\begin{figure}

{\centering \includegraphics{images/lab5_macro.png}

}

\caption{HeteroskedaticityV3 Macro}

\end{figure}

If you do not manage to install the macro. Do not panic! There is
another option without installation -- see below:

\hypertarget{running-the-macro-without-installation}{%
\section{Running the Macro without
Installation}\label{running-the-macro-without-installation}}

You can run the macro without having it installed on your machine.
Follow these steps:

\begin{itemize}
\item
  Go back to the download folder in your PC where you have downloaded
  several files from this page
  \url{https://github.com/ahmaddaryanto/Heteroskedasticity} (see
  Section~\ref{sec-install})
\item
  HeteroskedasticityV3.sps was included when you download the files from
  the github page.
\item
  Open the HeteroskedasticityV3.sps on your SPSS -- it will be opened as
  a syntax file,
\item
  Run the syntax file (highlight all the lines, run the selection
  button, i.e., the green button).
\item
  Next, open this file: \textbf{Runthemacro.sps}.
\item
  Change the DV and IVs according to your model specifications, and
\item
  Run the file.
\end{itemize}

\hypertarget{video-3}{%
\section{Video}\label{video-3}}

{[}Lecture Week 16 on Correlation and Regression{]}

\href{https://dtu-panopto.lancs.ac.uk/Panopto/Pages/Viewer.aspx?id=a8f3e3c0-35c3-477c-950f-afa3007195b2}{2023}

\href{https://dtu-panopto.lancs.ac.uk/Panopto/Pages/Viewer.aspx?id=074239e8-f79d-4ffb-8754-ac9400f4f00f}{Regression
with a categorical iv}

\href{https://dtu-panopto.lancs.ac.uk/Panopto/Pages/Viewer.aspx?id=5ba02b44-b4d1-4b70-ab45-acb900dc20a4}{Regression
diagnostics}



\end{document}
